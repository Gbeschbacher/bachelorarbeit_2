\section{Web-Components Praxisbeispiel}
\label{sec:6_Web_Components_Praxis}

In diesem Kapitel wird das Praxisprojekt, das für diese Arbeit entstanden ist, beschrieben. Der gesamte Quellcode ist auf \url{http://github.com/gbeschbacher/bachelorarbeit_2/} im Unterordner \lstinline|praxisprojekt| erreichbar. Es ist von Vorteil, wenn zu Beginn die Datei \glqq FeatureDetection.html\grqq\ aufgerufen und das Ergebnis angesehen wird. Diese Datei zeigt die Unterstützung sämtlicher Funktionen von Web-Components in der Browser-Konsole an. Ist die Unterstützung im verwendeten Browser nicht gegeben, können die Beispiele in den Unterordnern \lstinline|diagram| und \lstinline|menu| nicht angesehen werden. Diese beiden Beispiele bauen auf nativen Schnittstellen von Web-Components auf. Genaueres zur Entwicklung mit Hilfe von nativen Schnittstellen ist im Kapitel \ref{sec:6_WC_Pur} auf Seite \pageref{sec:6_WC_Pur} beschrieben.

Trotz vielleicht fehlender Unterstützung von nativen Funktionen des Browsers, kann ein Teil des Praxisprojekts ausgeführt und angesehen werden. Googles Polyfill namens Polymer garantiert die Unterstützung von \glqq Evergreen\grqq -Browsern. Genaueres zur Installierung und Verwendung von Polymer ist im Kapitel \ref{sec:6_WC_Polymer} auf Seite \pageref{sec:6_WC_Polymer} beschrieben.

Generell wurden zwei verschiedene Web-Komponenten einerseits mittels nativen Schnittstellen des Browsers und andererseits mit Polymer als Polyfill entwickelt.

Die erste entwickelte Komponente ist ein dynamisches Diagramm. Mit Hilfe von \glqq data-attributes\grqq\ können diverse optionale Einstellungen für diese Komponente festgelegt werden, wie beispielsweise das Intervall, in dem sich das Diagramm aktualisiert. Weiters kann die maximale Anzahl an sichtbaren Datenpunkte des Diagramms angegeben werden. Die Daten die im Diagramm gezeigt sind werden einfachheitshalber berechnet. Jeder Datenpunkt ist abhängig von seinem vorherigen und unterscheidet sich nur in einem vordefinierten Bereich. Die Komponente könnte ohne weiteres dahingehend ausgebaut werden, dass die Daten von einer externen Datenquelle entgegengenommen und visualisiert werden. Für die Visualisierung beziehungsweise der Erstellung des Diagramms wurde die frei verfügbare Bibliothek \lstinline|canvas.js|\footnote{Mehr Informationen zu canvas.js auf \href{http://canvasjs.com/}{http://canvasjs.com/}} benutzt.

Die zweite entwickelte Komponente ist eine \glqq Multi-Level\grqq -Menü Komponente, die auf dem von Codrops zur Verfügung gestellten Bibliotheken basiert\footnote{Mehr Informationen zu den Bibliotheken auf \href{https://github.com/codrops/MultiLevelPushMenu}{github}}. Diese Komponente rendert eine vordefinierte Menüstruktur, die sowohl Desktop- als auch Mobile-tauglich ist. Zur Zeit kann diese Komponente nur visuell angepasst werden, jedoch nicht dynamisch durch beispielsweise weitere Menüpunkte erweitert werden.

Durch die Entwicklung der beiden Komponenten wird festgestellt, inwiefern bereits komponentenbasierte Softwarearchitektur-Konzepte beziehungsweise komponentenbasierte Softwareentwicklung miteinbezogen werden. Diese Analyse wird zum einen mit nativen Web-Components und zum anderen mit Polymer erstellt. Das Resultat dieser Frage wird als Grundbasis für die Beantwortung der Forschungsfrage dienen.