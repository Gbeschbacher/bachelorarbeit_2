\section{Konklusio}
\label{sec:5_Konklusion}

Wie Komponenten im Web entwickelt werden können beziehungsweise welche Anforderungen sie gerecht werden müssen, wurde bis dato nicht standardisiert. Somit gab es mehrere Möglichkeiten, wie Plugins, Widgets, etc. entwickelt wurden. Jedoch waren die Interoperabilität zwischen den Komponente meist nicht gegeben, da es keine standardisierte Kommunikationssprache zwischen den Komponenten gab. Dieses Problem wird durch die Verwendung von Web-Components gelöst. Web-Components ist ein Sammelbegriff von fünf Konzepten. Wenn sämtliche Konzepte zusammen verwendet werden, ist die Interoperabilität und Wiederverwendbarkeit von Web-Komponenten gegeben. Um dies näher erläutern zu können werden folgend die Konzepte von Web-Components mit der Definition einer klassischen Softwarekomponente gegenübergestellt:

\begin{enumerate}

\litem{A component is a unit of independent deployment}\hfill \\
Hinsichtlich Web-Components wird dieser Punkt hauptsächlich durch HTML-Importe und Shadow-DOM  realisiert. Damit eine Komponente \glqq independent deployable\grqq\ sprich unabhängig auslieferbar ist, muss sie auch so konzipiert und entwickelt sein. Shadow-DOM erlaubt es, sämtliche Unterstrukturen eines Elements vollständig zu kapseln, sodass es keine Interferenzen zu anderen Komponenten geben kann. Um sie noch ausliefern zu können, werden HTML-Importe verwendet. Diese erlauben es, eine Komponente mit all ihren Abhängigkeiten als eine Datei auszuliefern.

\litem{A component is a unit of third-party composition} \hfill \\
Komponenten, die mit den Konzepten von Custom-Elements und Shadow-DOM von Web-Components umgesetzt wurden, sind zusammensetzbar. Durch die gemeinsame Verwendung dieser beiden Technologien können Autorinnen und Autoren von Komponenten klar definierte Schnittstellen entwickeln und Komponenten vollständig kapseln. Dies bedeutet, dass die entwickelten Komponenten nur über die zur Verfügung gestellten Schnittstellen verwendet werden können. Durch die Gegebenheit dieser klar definierten Grenze zwischen Innerem und Äußerem können keine Interferenzen mit anderen Komponenten entstehen. Somit ist auch die Interaktion zwischen dem System und der Komponente über die definierten Schnittstellen gegeben.

\litem{A component has no (externally observable) state}\hfill \\
Die Definition besagt, dass Originalkomponente nicht von Kopien ihrer selbst unterschiedlich sein dürfen. Web-Components bieten für diesen Punkt der Definition keinerlei Hilfestellungen und somit obliegt es der Autorin beziehungsweise dem Autor der Komponente, dies zu berücksichtigen.
Auch wenn externe Zustände erlaubt wären, die nicht zur Funktionalität der Komponente an sich beitragen, sind Web-Components in diesem Kontext nicht behilflich.

\end{enumerate}

Wenn die Konzepte von Web-Components und Polymer hinsichtlich klassischer Softwarearchitektur in Verbindung gebracht werden, entstehen Überschneidungen der Grundkonzepte. Somit werden die Konzepte bereits von klassischer Softwarearchitektur beeinflusst und nehmen auf die Konzepte dieser Rücksicht (siehe Kapitel \ref{sec:5_Vergleich_Architektur} auf Seite \pageref{sec:5_Vergleich_Architektur}).

Folgend werden nochmals kurz die Resultate der Vergleiche von Web-Components und Polymer mit diversen Aspekten der Softwarearchitektur beschrieben:

\begin{description}
\item[Vergleich hinsichtlich komponentenbasierter Softwarearchitektur] \hfill \\
Durch die Bereitstellung von vordefinierten Elementen bietet Polymer mehr Vorteile hinsichtlich komponentenbasierter Softwarearchitektur als native Web-Components.

\item[Vergleich hinsichtlich serviceorientierter Softwarearchitektur] \hfill \\
In Bezug auf die serviceorientierte Softwarearchitektur bieten weder native Web-Components, noch Polymer Hilfestellungen. Mit Kompositionen von Systemkomponenten, die bei SOA Dienste darstellen, können Software-Systeme aufgebaut werden. Dienste wiederum können auf lokal vorhandene Komponenten zurückgreifen.
\end{description}

Werden Web-Components und Polymer mit dem Konzept der komponentenbasierten Softwareentwicklung in Relation gesetzt, ergibt sich, dass Polymer mehr Vorteile hinsichtlich dieser Entwicklungsmethode bietet, als native Web-Components. Durch das Paradigma, das Polymer verfolgt (\glqq Alles ist eine Komponente und entwickelt sich mit dem Web mit \grqq ), entsteht ein Vorteil gegenüber nativen Web-Components der folgend erklärt wird: Auf Grund des Paradigmas stellt Polymer eine Vielzahl von vordefinierten Elementen zur Verfügung, die beliebig verwendet und erweitert werden können. Durch diese bereitgestellten Elemente verfolgt Polymer noch ein weiteres wichtiges Konzept von komponentenbasierter Softwareentwicklung: Systeme sollen auf Basis von Standardkomponenten aufbauen, anstatt bereits funktionierende Komponenten neu zu entwickeln.

Im Hinblick auf die zur Verfügung gestellten Funktionen von nativen Web-Components und Polymer ist Polymer im Vorteil. Folgend werden die Vorteile von Polymer gegenüber Web-Components kurz erläutert:
\begin{itemize}
\item Zusätzlich zu normalen Templates stellt Polymer eine Vielzahl von Template-Besonderheiten wie beispielsweise Template-Bindings bereit. Diese Besonderheiten vereinfachen beziehungsweise erweitern die Funktionalitäten von Templates um einiges.
\item Durch die zwei neuen Lebenszyklus-Callback-Methoden von Custom-Elements bei Polymer ist es möglich, besser auf spezielle Ereignisse zu reagieren.
\item Durch die limitierte Kapselung von Shadow-DOM sind sämtliche Inhalte im Shadow-DOM von Suchmaschinen, Screen-Readern, Browser-Extensions, etc. erreichbar. Dies kann sowohl ein Vor-, als auch Nachteil sein. Somit obliegt die Entscheidung hierbei bei der Leserin beziehungsweise dem Leser des Autors.
\item Die Bereitstellung von vordefinierten Elementen bringt einige Vorteile hinsichtlich komponentenbasierter Softwarearchitektur und komponentenbasierter Softwareentwicklung mit sich.
\item Die Unterstützung der Spezifikationen von Web-Animations und Pointer-Events, sowie die Bereitstellung von Polyfills dieser Konzepte, vereinfachen die Entwicklung von Cross-Browser-Applikationen um einiges.
\end{itemize}

Ein weiterer wichtiger Vorteil von Polymer gegenüber Web-Components ergibt sich, wenn die Unterstützung hhinsichtlich der Entwicklung und insichtlich der Browser verglichen wird. Polymer wird bereits von einigen Browsern zu 100\% unterstützt. Auch ist die Unterstützung seitens der Entwicklerinnen und Entwickler von Polymer sehr gut. Durch die Aktivität und Präsenz der Entwicklerinnen und Entwickler von Polymer in einigen Foren steigt die Attraktivität des Frameworks.