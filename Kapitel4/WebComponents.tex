\subsection{Programmierung von Web-Components nach dem W3C Standard}
\label{sec:4_WC_Pur}

Dieses Subkapitel baut auf den bereits vorher beschriebenen Grundlagen von Web-Components auf (siehe Kapitel \ref{sec:3_W3C} auf Seite \pageref{sec:3_W3C}). Des Weiteren wird zuerst die Diagramm-Komponente beschrieben und daraufhin die Menü-Komponente.

\textbf{Diagramm-Komponente}

Im Hauptordner der Diagramm-Komponente befinden sich 3 HTML-Dateien. Diese Dateien unterscheiden sich nur gering voneinander. Sie repräsentieren verschiedene Art und Weißen wie man die Technologien von Web-Components zusammen verwenden kann. Folgend wird jede Datei einzeln besprochen.

\begin{enumerate}
\litem{\lstinline|main.html|-Datei} \hfill \\
Diese Datei stellt zwei Diagramm-Komponenten dar, die voneinander unabhängig sind. Mit Hilfe der \lstinline|document-register()|-Methode wurde das Element \lstinline|<chart-live>| registriert und mittels der verfügbaren Lebenszyklus-Methoden wird das Element erstellt und entwickelt. In der \lstinline|createdCallback|-Methode werden sämtliche Initialisierungswerte, die für das Diagramm notwendig sind, gesetzt. Zeile 10 zeigt dann die Erstellung des eigentlichen Diagramms mit Hilfe von \lstinline|CanvasJS|. Im \lstinline|attachedCallback| wird das Intervall für die Aktualisierung des Diagramms gesetzt und aufgerufen. In der \lstinline|updateChart|-Methode werden wie bereits in der Einleitung geklärt die Daten für das Diagramm einfachheitshalber selbst berechnet. Code-Beispiel \ref{lst:4_mainhtml} auf Seite \pageref{lst:4_mainhtml} ist dem aktuellen Quellcode entnommen und zeigt einen Ausschnitt der entwickelten Methoden, welche in der \lstinline|main.html|-Datei zu sehen sind.

\lstinputlisting[language=JavaScript, firstline=27, lastline=73, caption={main.html}, label={lst:4_mainhtml}]{./praxisprojekt/diagram/main.html}

\litem{\lstinline|main2.html|-Datei} \hfill \\
Diese Datei unterscheidet sich vom Grundkonzept von der \lstinline|main.html|-Datei. Die gesamte Komponente wurde ausgelagert und wird nur noch über das in Code-Beispiel \ref{lst:4_main2html} auf Seite \pageref{lst:4_main2html} gezeigte JavaScript verwendet. Das Diagramm, welches in der Datei \lstinline|permChart.html| definiert wird, wird durch einen HTML-Import geladen. Die \lstinline|permChart.html|-Datei unterscheidet sich von den Diagramm-Funktionen nicht im Vergleich zur \lstinline|main.html|-Datei. Der grundlegende Unterschied ist der Aufbau der Komponente. Code-Beispiel \ref{lst:4_permCharthtml} auf Seite \pageref{lst:4_permCharthtml} zeigt, dass bei diesem Beispiel bereits ein \lstinline|<template> und <content>|-Element verwendet wird. Diese beiden Elemente sind wichtig, um das Shadow-DOM in diesem Fall richtig benutzen zu können. Code-Beispiel \ref{lst:4_permCharthtml2} auf Seite \pageref{lst:4_permCharthtml2} stellt dies genauer dar. Nur statische Elemente des Diagramms befinden sich im Shadow-DOM, da sich das Diagramm beispielsweise über die externe Bibliothek \lstinline|CanvasJS| aktualisiert und somit es noch von außen erreichbar sein muss.

\lstinputlisting[language=HTML, firstline=2, lastline=8, caption={permChart.html-Aufbau}, label={lst:4_permCharthtml}]{./praxisprojekt/diagram/imports/permChart.html}
\lstinputlisting[language=JavaScript, firstline=61, lastline=63, caption={permChart.html-Verwendung}, label={lst:4_permCharthtml2}]{./praxisprojekt/diagram/imports/permChart.html}
\lstinputlisting[language=JavaScript, firstline=25, lastline=27, caption={main2.html-Verwendung von der in permChart.html definierten Komponente}, label={lst:4_main2html}]{./praxisprojekt/diagram/main2.html}

\litem{\lstinline|main3.html|-Datei} \hfill \\
Dieses Beispiel ist sehr ähnlich zu Beispiel 1 aus Datei \lstinline|main.html|. Der Unterschied liegt darin, dass die Datei \lstinline|main3.html| die Diagramm-Komponente vollständig importiert und dieser Import sämtliche Notwendigkeiten wie beispielsweise die Registrierung des benutzerdefinierten Elements \lstinline|<chart-live>| übernimmt. Somit kann das Element verwendet werden, ohne selbst ein Element registrieren zu müssen.
\end{enumerate}

\textbf{Menü-Komponente}

Die Menü-Komponente ist sehr einfach und ähnlich zur Datei \lstinline|main3.html| der Diagramm-Komponente aufgebaut. In der Hauptdatei der Menükomponente wird lediglich die \lstinline|<push-menu>|-Komponente per HTML-Import geladen. Die Registrierung und Darstellung, sprich Markup des Elements wird in eine externe Datei (pushMenu.html) ausgelagert, um so die Wiederverwertbarkeit garantieren zu können, ohne Code duplizieren zu müssen.

Die \lstinline|pushMenu|-Datei beginnt mit einem Import der gebrauchten CSS- und JavaScript-Dateien von Codrops (siehe Code-Beispiel \ref{lst:4_pushMenuhtml} auf Seite \pageref{lst:4_pushMenuhtml}). Weiters folgt das HTML-Markup des Menüs, wie in Code-Beispiel \ref{lst:4_pushMenuhtml3} auf Seite \pageref{lst:4_pushMenuhtml3} zu sehen ist. Es wird aus Platzgründen nur der Anfang des Templates gezeigt. Die grundlegende Initialisierung des Menüs liegt ausschließlich in der \lstinline|attachedCallback|-Lebenszyklus Methode, wie in Code-Beispiel \ref{lst:4_pushMenuhtml2} auf Seite \pageref{lst:4_pushMenuhtml2} zu sehen ist.

Ein wichtiger Punkt bei dieser Komponente was Wiederverwertbarkeit anbelangt ist, dass sie zur Zeit nur statisch entwickelt wurde. Dies bedeutet, dass sie keinerlei Anpassungsmöglichkeit hinsichtlich der Menüstruktur für andere Projekte bietet, was diese Komponente nicht zu einem idealen Beispiel macht, jedoch aus Zeitgründen nicht realisiert werden konnte.\todo{check ob umschreiben}

\lstinputlisting[language=HTML, firstline=1, lastline=1, caption={pushMenu.html - Import der Bibliotheken von Codrops}, label={lst:4_pushMenuhtml}]{./praxisprojekt/menu/imports/pushMenu.html}

\lstinputlisting[language=HTML, firstline=3, lastline=5, caption={pushMenu.html - Beginn des Templates}, label={lst:4_pushMenuhtml3}]{./praxisprojekt/menu/imports/pushMenu.html}

\lstinputlisting[language=JavaScript, firstline=62, lastline=66, caption={pushMenu.html - attachedCallback-Methode des Menüs}, label={lst:4_pushMenuhtml2}]{./praxisprojekt/menu/imports/pushMenu.html}
