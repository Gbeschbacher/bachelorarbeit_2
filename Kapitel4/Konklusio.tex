\subsection{Konklusio}
\label{sec:4_Konklusion}

In diesem Kapitel wurde das Framework namens \glqq Polymer\grqq\ näher veranschaulicht. Dieses Framework bietet Polyfills für vier der fünf Konzepte von \glqq Web-Components\grqq . Neben den Polyfills für Web-Components werden weitere Funktionalitäten wie beispielsweise \glqq Pointer-Events\grqq , \glqq Web-Animations\grqq und eine Vielzahl vordefinierter Elemente zur Verfügung gestellt.

Die Unterstützung von Polymer hinsichtlich der Browser ist sehr gut. Beinahe 100\% der Browser unterstützen sämtliche Polyfills von Polymer. Auch ist die Unterstützung seitens der Entwicklerinnen und Entwickler nennenswert. Das Polymer-Team versucht sämtliche Fragen von Benutzerinnen und Benutzer beantworten zu können und ist somit auch auf einer Vielzahl von Foren präsent.

Wichtig zu nennen ist jedoch, dass sämtliche Polyfills sowohl einzeln, als auch gemeinsam verwendet werden können. Wenn alle vier Polyfills gemeinsam verwendet werden, müssen nicht alle Polyfills extra inkludiert werden. Die Entwicklung von Komponenten mit Hilfe aller Polyfills wird in Kapitel \ref{sec:6_WC_Polymer} auf Seite \pageref{sec:6_WC_Polymer} näher erläutert.
