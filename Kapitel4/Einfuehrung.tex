In diesem Kapitel wird das Praxisprojekt, das für diese Arbeit entstanden ist, beschrieben. Der gesamte Quellcode dazu ist unter \url{http://github.com/gbeschbacher/bachelorarbeit_2/} im Unterordner \lstinline|praxisprojekt| erreichbar. Es ist von Vorteil, wenn zu Beginn die Datei \glqq FeatureDetection.html\grqq\ aufgerufen und das Ergebnis angesehen wird. Dadurch sieht man, ob der verwendete Browser sämtliche Funktionen von Web-Components unterstützt. Wenn dies nicht der Fall ist, wird man die Beispiele in den Unterordnern \lstinline|diagram| und \lstinline|menu| nicht ansehen können, denn sie bauen auf den nativen Schnittstellen von Web-Components auf. Genaueres zu der Entwicklung mit den nativen Schnittstellen ist im Kapitel \ref{sec:4_WC_Pur} auf Seite \pageref{sec:4_WC_Pur} beschrieben.

Trotz vielleicht fehlender Unterstützung von nativen Funktionen des Browsers, welche mit Hilfe der FeatureDetection-Datei eingesehen werden können, kann ein Teil des Praxisprojekts ausgeführt und angesehen werden. Genaueres zur Installierung und Verwendung von Polymer ist im Kapitel \ref{sec:4_WC_Polymer} auf Seite \pageref{sec:4_WC_Polymer} einzusehen.

Generell wurde versucht zwei verschiedene Web-Komponenten einerseits mittels nativen Schnittstellen des Browsers und andererseits mit Polymer und Browser-Polyfills zu entwickeln.

Die erste Komponente ist ein dynamisches Diagramm. Mit Hilfe von \glqq data-attributes\grqq\ können diverse optionale Daten für diese Komponente festgelegt werden, wie beispielsweise das Intervall, in dem die Komponente sich aktualisieren sollte. Weiters kann die Länge des Diagramms angegeben werden. Die Daten die im Diagramm angezeigt werden, werden einfachheitshalber berechnet. Jeder Datenpunkt ist abhängig von seinem vorherigen und unterscheidet sich nur in einem vordefinierten Bereich. Die Komponente könnte ohne weiteres dahingehend ausgebaut werden, dass die Daten von einer externen Datenquelle entgegengenommen und visualisiert werden. Für die Visualisierung beziehungsweise der Erstellung des Diagramms wurde die frei verfügbare Bibliothek canvas.js\footnote{Mehr Informationen zu canvas.js auf \href{http://canvasjs.com/}{http://canvasjs.com/}} benutzt.

Die zweite Komponente ist eine \glqq Multi-Level\grqq -Menü Komponente, die auf dem von Codrops zur Verfügung gestellten Bibliotheken basiert\footnote{Mehr Informationen zu den Bibliotheken auf \href{https://github.com/codrops/MultiLevelPushMenu}{github}}. Diese Komponente rendert eine vordefinierte Menüstruktur, die sowohl Desktop- als auch Mobile-tauglich ist. Zur Zeit kann diese Komponente nur visuell angepasst werden, jedoch nicht dynamisch durch beispielsweise weitere Menüpunkte erweitert werden.

Durch Entwicklung der beiden Komponenten soll festgestellt werden, inwiefern bereits komponentenbasierte Softwarearchitektur-Konzepte berücksichtigt werden können beziehungsweise inwiefern Google-Polymer dabei Vor- oder Nachteile bietet.