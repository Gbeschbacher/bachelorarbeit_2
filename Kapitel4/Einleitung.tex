\section{Google Polymer}
\label{sec:4_Polymer}

In allgemeinen Worten beschrieben ist Google-Polymer ein Framework, das sämtliche Funktionen von Web-Components mit Hilfe eines Polyfills zur Verfügung stellt. Ein Polyfill ist ein Browser-Fallback, um Funktionen, die in modernen Browsern verfügbar sind, auch in alten Browsern verfügbar zu machen. Polymer verfolgt den Ansatz, dass alles was entwickelt wird, eine Web-Komponente ist und sich mit der Entwicklung des Webs weiterentwickelt.

In diesem Kapitel wird folgend zuerst die Architektur hinter dem Polymer-Framework erklärt. Danach werden die einzelnen Polyfills von Polymer näher erläutert und spezielle Eigenschaften des Frameworks kurz beschrieben. Polymer stellt zurzeit für vier der fünf Spezifikationen von Web-Components einen Polyfill bereit. Kapitel \ref{sec:4_Polymer_PTemplates}, \ref{sec:4_Polymer_PElements}, \ref{sec:4_Polymer_PShadowDOM} und \ref{sec:4_Polymer_PHTMLImports} folgen der Annahme, dass die Polyfills nicht miteinander verwendet werden. Wenn sämtliche Polyfills, die Polymer bereitstellt, verwendet werden möchten, stellt Polymer zwei Dateien zur Verfügung. Welche Dateien das sind und wie die Polyfills dann verwendet werden wird in Kapitel \ref{sec:6_WC_Polymer} auf Seite \pageref{sec:6_WC_Polymer} an Hand von zwei entwickelten Komponenten beschrieben.

Abschließend wird die Unterstützung von Polymer seitens der Entwickler und seitens der Browser noch näher erläutert.








