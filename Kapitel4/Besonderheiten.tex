\subsection{Besonderheiten von Polymer}
\label{sec:4_Polymer_Besonderheiten}

Zusätzlich zu den Web-Components stellt Polymer noch zwei weitere Konzepte bereit:
\begin{enumerate}
\litem{Pointer-Events} \hfill \\
Mouse- und Touch-Events sind zwei fundamental unterschiedliche Events in Browsers. Aktuelle Plattformen, die Touch-Events implementiert haben, haben auch Mouse-Events implementiert, um die Abwärtskompatibilität gewährleisten zu können. Jedoch werden nur Mouse-Events nur bedingt gefeuert und besitzen auch eine andere Semantik. Die nachstehende Liste zeigt einige Beispiele für die bedingte Feuerung von Mouse-Events:
\begin{itemize}
\item Mouse-Events werden nur gefeuert, wenn die Touch-Sequenz endet.
\item Mouse-Events werden nicht von Elementen ohne Click-Event-Handler gefeuert.
\item Click-Events werden nicht gefeuert, wenn der Inhalt einer Seite sich durch ein Mouse-Move-Event oder Mouse-Over-Event ändert.
\item Click-Events werden mit einer Verzögerung von 300ms nach Ende einer Touch-Sequenz gefeuert.
\end{itemize}

Auch werden Touch-Events nur zu dem Element gesendet, dass das Touch-Start-Event bekommt. Dies ist grundsätzlich unterschiedlich zu Mouse-Events. Mouse-Events werden auf das Element, dass sich unter der Maus befindet, gesendet. Um hier eine Gemeinsamkeit erreichen zu können, werden Touch-Events mit Hilfe von \lstinline|document.elementFromPoint| neu kalibriert.

Um solche Inkompatibilitäten vorzubeugen, stellt Polymer Pointer-Events bereit.

\litem{Web-Animations} \hfill \\
Web-Animations ist eine neue Spezifikation für animierten Inhalt im Internet. Es wird als W3C Spezifikation als Teil der CSS- und SVG-Working-Group entwickelt. Diese Spezifikation thematisiert die Mängel der bereits vorhandenen Spezifikationen zu CSS- und SVG-Animationen. Web-Animations sollen des Weiteren die zugrunde liegenden Implementierungen von CSS-Transitions, CSS-Animations und SVG-Animations ersetzen. Dies würde dabei helfen, den Code für Unterstützung von Animationen im Web gering zu halten und die verschiedenen Animations-Spezifikationen interoperabel zu machen.
\end{enumerate}

Polymer bietet neben den bereits erläuterten Polyfills und Besonderheiten auch die Möglichkeit vordefinierte Elemente zu verwenden. Ein Beispiel für ein vordefiniertes Element wäre das \lstinline|<polymer-ajax>|-Element. Es versucht einen Standard für Entwicklerinnen und Entwickler bereitzustellen, um Ajax-Requests zu erstellen beziehungsweise abzuwickeln. Dieses Element ist ähnlich zu folgender Funktion von jQuery: \lstinline{$.ajax()}\footnote{Mehr Information zur jQuery.ajax-Funktion unter \href{http://api.jquery.com/jQuery.ajax/}{http://api.jquery.com/jQuery.ajax/}}. Der Unterschied zwischen den beiden Möglichkeiten einen Ajax-Request abzuwickeln ist, dass die \lstinline{$.ajax()}-Methode Abhängigkeiten besitzt, wohingegen die \lstinline|<polymer-ajax>|-Methode nur pures JavaScript benutzt und somit nur Polymer als Abhängigkeit besitzt.

Wie Custom-Elements mittels Polymer entwickelt werden können und welche Coding-Conventions dieses Framework beinhaltet, wird in Kapitel \ref{sec:6_WC_Polymer} auf Seite \pageref{sec:6_WC_Polymer} genauer erläutert.

