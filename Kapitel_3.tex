\section{Web-Components}
\label{sec:3_Web_Components}

\todo[inline]{HTML Templates W3C Draft (18.März.2014) https://dvcs.w3.org/hg/webcomponents/raw-file/tip/spec/templates/index.html}
%http://www.w3.org/TR/html5/scripting-1.html#the-template-element

Um Web-Components besser verstehen zu können, wird in diesem Kapitel zu Beginn eine kurze Übersicht über die Geschichte von Web-Bibliotheken gezeigt.

\begin{description}
\item[2005] Veröffentlichung von Dojo Toolkit\footnote{Mehr Information zu Dojo Toolkit unter \href{http://dojotoolkit.org/}{http://dojotoolkit.org/}} mit der innovativen Idee von Widgets. Mit ein paar Zeilen Code konnten Entwickler komplexe Elemente, wie beispielsweise einen Graph oder eine Dialog-Box in ihrer Website hinzufügen.
\item[2006] jQuery\footnote{Mehr Information zu jQuery unter \href{http://jquery.com/}{http://jquery.com/}} stellt Entwicklern die Funktion zur Verfügung Plugins zu entwickeln, die später wiederverwendet werden können.
\item[2008] Veröffentlichung von jQuery UI\footnote{Mehr Information zu jQuery UI unter \href{http://jqueryui.com/}{http://jqueryui.com/}}, was vordefinierte Widgets und Effekte mit sich bringt.
\item[2009] Erstveröffentlichung von AngularJS\footnote{Mehr Information zu AngularJS unter \href{http://angularjs.org/}{http://angularjs.org/}}, ein Framework mit Direktiven.
\item[2011] Erstveröffentlichung von React\footnote{Mehr Information zu Facebook React unter \href{http://facebook.github.io/react/}{http://facebook.github.io/react/}}. Diese Bibliothek gibt den Entwicklern die Fähigkeit, das User Interface ihrer Website zu bauen, ohne dabei auf andere Frameworks, die auf der Seite benutzt werden, achten zu müssen
\item[2013] Veröffentlichung des Entwurfs von Web-Components, jedoch mit schlechter Browser Unterstützung
\end{description}

Mit der Veröffentlichung von Dojo Toolkit sagen Entwickler die Vorteile von wiederverwendbaren Modulen. Wenn man zurzeit Plugins auf einer Website erwähnt, denken die meisten Entwickler von jQuery Plugins, da sie beinahe überall Verwendung finden und ein großes Spektrum von Funktionen bieten. Mit den Veröffentlichungen von AngularJS und React wurde gezeigt, in welche Richtung sich Web-Anwendungen bewegen. Sie zeigen, dass es nicht nur um visuelle Elemente geht, sondern auch um Elemente, die eine komplexe Logik besitzen.




Ähnlich zu HTML5 ist Web-Components ein Sammelbegriff für mehrere Features:
\begin{description}
\item[Shadow DOM (ausführliche Erklärung siehe Kapitel \ref{sec:3_WC_Shadow_DOM} auf Seite \pageref{sec:3_WC_Shadow_DOM})] erlaubt es das DOM und CSS zu kapseln
\item[HTML Templates (ausführliche Erklärung siehe Kapitel \ref{sec:3_WC_Templates} auf Seite \pageref{sec:3_WC_Templates})] sind ein Weg, um den DOM zu klonen und somit den Klon wiederzuverwenden
\item[Custom Elements (ausführliche Erklärung siehe Kapitel \ref{sec:3_WC_Elements} auf Seite \pageref{sec:3_WC_Elements})] können einerseits neue Elemente definieren, oder bereits bestehende Elemente erweitern. Dies bedeutet, dass ein Entwickler beispielsweise den HTMl \lstinline|<input>|-Tag dahingehend erweitern kann, dass dieser nur das Format von Kreditkartennummern unterstützt. Ein Beispiel für die Definition eines neuen Elements wäre ein Element, dass sämtliche Felder, die für die Bezahlung mit einer Kreditkarte notwendig sind, bereitstellt.
\item[HTML Imports (ausführliche Erklärung siehe Kapitel \ref{sec:3_WC_Imports} auf Seite \pageref{sec:3_WC_Imports})] sind dazu da, um externe HTML-Dateien in die bestehende Website zu integrieren, ohne dabei den Code kopieren zu müssen. Sie können beispielsweise dazu verwendet werden, um Web-Components in eine Website zu integrieren.
\item[Decorators (ausführliche Erklärung siehe Kapitel \ref{sec:3_WC_Decorators} auf Seite \pageref{sec:3_WC_Decorators})] sind Elemente, die nach dem \glqq Decorator-pattern\grqq\ benannt sind. Durch dieses Pattern ist es möglich Elemente um zusätzliche Funktionalitäten zur Laufzeit erweitern zu können.
\end{description}

Obwohl \glqq Web-Components\grqq\ für viele Entwickler noch kein Begriff ist, wird es bereits vom Browser automatisch verwendet. Beispiele dafür sind der Datepicker oder das \lstinline|<video>|-Elemen. Abbildung \ref{fig:3_Datepicker_Visuals} auf Seite \pageref{fig:3_Datepicker_Visuals} zeigt die Datepicker-Komponente und Abbildung \ref{fig:3_Datepicker_Source} auf Seite \pageref{fig:3_Datepicker_Source} zeigt den dazugehörigen Source-Code. Dieser Code zeigt, dass sämtliche Kontrollbuttons des Datepickers vor dem Entwickler \glqq versteckt\grqq , also im Shadow DOM liegen.


\begin{figure}[h]
\centering
\includegraphics[]{images/datepicker.jpg}
\missingfigure{Datepicker (Visual)}
\caption[
Beispiel von Web-Components im Browser an Hand von dem Datepicker, Urldate: 04.2014 \newline
\small\texttt{https://s3.amazonaws.com/infinum.web.production/repository\_items/files/000/000/238/original/datepicker.jpg}
]{Beispiel von Web-Components im Browser an Hand von dem Datepicker}
\label{fig:3_Datepicker_Visuals}
\end{figure}

\begin{figure}[h]
\centering
\includegraphics[height=5.0cm]{images/datepicker_shadow_dom.jpg}
\caption[
Beispiel von Web-Components im Browser an Hand von dem Datepicker, Urldate: 04.2014 \newline
\small\texttt{https://s3.amazonaws.com/infinum.web.production/repository\_items/files/000/000/236/original/datepicker\_shadow\_dom.jpg}
]{Beispiel von Web-Components im Browser an Hand von dem Datepicker}
\label{fig:3_Datepicker_Source}
\end{figure}

\textbf{Warum Web-Components?}\\
Javascript Widgets und Plugins sind fragmentiert, weil sie auf diversen unterschiedlichen Bibliotheken und Frameworks basieren, die möglicherweise nicht miteinander funktionieren. Web-Components versuchen einen gewissen Standard in Widgets und Plugins zu bringen. Das Problem der nicht miteinander funktionierenden Plugins versucht Web-Components mit Kapselung zu lösen. Durch die Lösung dieses Problems ist die Wiederverwendbarkeit von Komponenten garantiert, da es sämtliche Interferenzen\todo{richtiges Wort?} zwischen Plugins löst. Web-Components können des Weiteren viel mehr als nur UI-Komponenten sein. Eine Bibliothek könnte bereits eine Komponente darstellen, die eine gewisse Funktionalität bereitstellt.

\textbf{Unterstützung von Web-Components}\\
Zur Zeit ist die Hauptproblem von Web-Components die mangelhafte Browserunterstützung. Kein einziger Browser unterstützt diesen Standard zu 100\%. Es gibt bereits mehrere Möglichkeiten beziehungsweise Polyfills\footnote{Ein Polyfill ist ein Browser-Fallback, um Funktionen, die in modernen Browsern verfügbar sind, auch in alten Browsern verfügbar zu machen.}, um dennoch Web-Components nutzen zu können. Beispiele hierfür sind:
\begin{itemize}
\item Polyfill-Webcomponents\footnote{Mehr Information zu Polyfill-Webcomponents unter \href{http://github.com/timoxley/polyfill-webcomponents}{http://github.com/timoxley/polyfill-webcomponents}}
\item Polymer-Project\footnote{Mehr Information zu Polymer unter \href{http://www.polymer-project.org/}{http://www.polymer-project.org/}}
\item X-tags\footnote{Mehr Information zu X-tags unter \href{http://x-tags.org/}{http://x-tags.org/}}
\end{itemize}
Obwohl es mehrere Polyfills bezüglich Web-Componnts gibt, ist es egal, auf welcher Basis man seine Web-Components programmiert, denn durch die Standardisierung ist die Interkompatibilität der einzelnen Komponenten gegeben. Diese Arbeit beschränkt sich hauptsächlich auf die Entwicklung von Web-Components mit Hilfe des Standards beziehungsweise der Polyfill-Bibliothek Polymer.\\
Durch die Benutzung von dem Polyfill Polymer funktionieren Web-Component in allen \glqq evergreen\grqq -Browsern\footnote{Ein \glqq Evergreen\grqq -Browser ist ein Web-Browser, der sich automatisch beim Start updatet.} und Internet-Explorer 10 und neuer. Des Weiteren funktionieren sie dadurch auf mobilen Endgeräten, wo iOS6+, Chrome Mobile, Firefox Mobile, oder Android 4.4 oder höher vorhanden ist. Auch mit Hilfe von Polyfill bieten sowohl der Internet-Explorer 9 und niedriger, als auch Android-Browser 4.3 oder niedriger keine Unterstützung für Web-Components. Dies bedeutet, dass Web-Components zur Zeit noch für die Verwendung im Web bereit sind, außer man hat als Zielgruppe ausschließlich eine Plattform, die unterstützt wird.

\textbf{Web-Components Alternativen}
Die folgenden Alternativen von Web-Components benutzen ähnliche Pattern um die gewünschte DOM-Abstrahierung zu erreichen.
\begin{description}
\item[React] benutzt seinen eigenen \glqq Virtual DOM\grqq und versucht in keinster Hinsicht Web-Components zu simulieren. Folglich ist die Browser-Unterstützung von React besser, als jene der Web-Components. Ab Internet-Explorer 8 werden sämtliche Browser vollständig unterstützt. Zur Zeit wird diese Technologie in Facebooks und Instagrams Kommentarsystemen eingesetzt.
\item[AngularJS] besitzt diverse Interferenzen zu Web-Components, jedoch versucht auch diese Technologie nicht Web-Components zu simulieren, um bessere Browser-Unterstützung zu bieten (Internet Explorer 8+). Die genaue Unterscheidung bezüglich AngularJS-Direktiven und Web-Components werden in Kapitel \ref{sec:3_Polymer} auf Seite \pageref{sec:3_Polymer} erklärt.
\end{description}


\subsection{W3C Web-Components Standard}
\label{sec:3_W3C}

\subsubsection{Templates}
\label{sec:3_WC_Templates}
Das folgende Kapitel basiert ausschließlich auf der Einführung zu Web-Components aus dem W3C \citereset \autocite{Glakov.2013}.
Laut W3C sind Templates
\begin{quote}
\glqq
  a method for declaring inert DOM subtress in HTML and manipulating them to instantiate document fragments with identical contents.
\grqq
\end{quote}
Somit sind Templates eine Methode um inaktive DOM-Subtrees im HTML zu deklarieren und manipulieren, um so sämtliche identische Dokumentfragmente mit identischem Inhalt zu instanziieren.\\
In Web-Applikationen muss man oft den gleichen Subtree von Elementen öfters benutzen, mit den passenden Inhalt füllen und es zum Maintree hinzufügen. Man hat zum Beispiel eine Liste von Artikel, die man mit mehreren \lstinline|<li>| -Tags in das Dokument einfügen will. Des Weiteren kann jeder \lstinline|<li>|-Tag weitere Elemente, wie beispielsweise einen Link, ein Bild, einen Paragraphen, etc., enthalten. Bis jetzt bot HTML keine native Möglichkeit an, eine solche Aufgabenstellung zu lösen. Mehrere Beispiele, wie Entwickler diese Aufgabe lösten, sind:
\begin{enumerate}
\litem{Versteckte Elemente}
Dies galt unter den Entwicklern als die einfachste, aber auch als die ineffizienteste Methode. Man gab sein Markup irgendwo in das DOM (meist mit der CSS-Eigenschaft \lstinline|display: none;| am Parent-Element des Markups) und wann immer es gebraucht wurde, wurde es geklont, mit diversen Daten gefüllt und schlussendlich in das DOM eingefügt. Dies beinhaltet diverse Nachteile. Sämtliche Ressourcen, die im Markup verwendet wurden, sind bei Seitenaufruf geladen worden. Die Gesamtperformanz des Browsers wurde durch die vielzählige DOM-Traversierung beeinträchtigt.
\litem{String-Templates}
Man versuchte die Probleme der Methode mit \glqq versteckten Elementen\grqq\ zu beseitigen, indem man in einem \lstinline|<script>|-Tag ein Template mit einem String definiert. Dadurch, dass diese Methode hinsichtlich Laufzeit-Parsen von \lstinline|innerHtml| geht, können XSS-Attacks ermöglicht werden \todo{Information von: http://ejohn.org/blog/javascript-micro-templating}. Folglich wird ein Beispiel gezeigt, wo ein Template mit Hilfe eines Strings in einem \lstinline|<script>|-Tag definiert wird:
\begin{lstlisting}[language=HTML, caption=String-Template]
  <script type="text/html" id="test_Showcase">
  </script>
\end{lstlisting}
\end{enumerate}
Folgend wird an Hand eines Beispiels, wo eine Liste von Autos benutzt wird, der neue Standard zu Templates erläutert werden:\\

\begin{lstlisting}[language=HTML, caption={Web-Components Template-Standard}, label={lst:3_Templates}, escapeinside={@}{@}]
<template id="carTemplate">
  <li>
    <span class="carBrand"></span>
    <span class="carName"></span>
  </li>
</template>
\end{lstlisting}

Ein Template, wie das aus der Listing \ref{lst:3_Templates} auf Seite \pageref{lst:3_Templates}, kann sowohl im \lstinline|<head>|- als auch im \lstinline|<body>|-Tag definiert werden. Das Template, inklusive Subtree, ist inaktiv. Dies bedeutet, dass wenn sich ein \lstinline|<img>|-Tag mit einer validen Quelle in diesem Template befinden würde, würde der Browser dieses Bild nicht laden. Des Weiteren ist es nicht möglich via JavaScript ein Element des Templates zu selektieren.
\begin{lstlisting}[language=JavaScript, caption={Beispiel-Selektor eines Elements in einem Template, das nicht aktiven DOM ist}, label={lst:3_Selector_Example}]
  document.querySelectorAll('.carBrand').length; // length ist 0
\end{lstlisting}

\begin{figure}[h]
\centering
%\includegraphics[height=5.0cm]{images/Julia-Fractal.png}
\missingfigure{DOM of Template (Documentfragment)}
\caption[
Visualisierung des DOM eines inaktiven Templates, Urldate: 04.2014 \newline
%\small\texttt{http://www.prevent-default.com/wp-content/uploads/2013/04/document-fragment-300x132.png}
]{Visualisierung des DOM eines inaktiven Templates}
\label{fig:3_inactive_Template_DOM}
\end{figure}

In Abbildung \ref{fig:3_inactive_Template_DOM} auf Seite \pageref{fig:3_inactive_Template_DOM} wird gezeigt, dass das Template ein Dokument-Fragment ist. Dies bedeutet dass es ein eigenständiges Dokument ist und unabhängig vom ursprünglichen Dokument existiert. Folglich bedeutet dies, dass sämtliche \lstinline|<script>, <form>, <img>|-Tags etc. nicht verwendet werden.

\begin{lstlisting}[language=JavaScript, caption={Verwendung des Templates \ref{lst:3_Templates} auf Seite \pageref{lst:3_Templates}}, label={lst:3_Templates_Verwendung}, escapeinside={@}{@}]
@\label{lst:3_Templates_Verwendung_1}@var template = document.getElementById('carTemplate');
template.content.querySelector(".carBrand").length; // length ist 1

@\label{lst:3_Templates_Verwendung_2}@var car = template.content.cloneNode(true);
car.querySelector(".carBrand").innerHTML = "Seat";
car.querySelector(".carName").innerHTML = "Ibiza";

@\label{lst:3_Templates_Verwendung_3}@document.getElementById("carList").appendChild(car);
\end{lstlisting}
Listing \ref{lst:3_Templates_Verwendung} auf Seite \pageref{lst:3_Templates_Verwendung} basiert auf dem in Listing \ref{lst:3_Templates} auf Seite \pageref{lst:3_Templates} definierten Template. Zuerst wird sich in Zeile \ref{lst:3_Templates_Verwendung_1} der Listing \ref{lst:3_Templates_Verwendung} das vorher definierte Template in die Variable \lstinline|template| geholt. Daraufhin wird der gesamte Knoten in Zeile \ref{lst:3_Templates_Verwendung_2} mit Hilfe einer \lstinline|deep-copy| geklont und des Weiteren mit Daten befüllt. Damit das mit Daten befüllte Listenelement auch sichtbar wird, wird es in Zeile \ref{lst:3_Templates_Verwendung_3} in das aktive DOM eingefügt.


\subsubsection{Decorators}
\label{sec:3_WC_Decorators}

Das folgende Kapitel basiert ausschließlich auf der Einführung zu Web-Components aus dem W3C \citereset \autocite[siehe][]{CooneyGlazkov.2013} und auf dem Artikel \glqq Decorators - NextGen Markup pt.2\grqq\ \citereset \autocite[siehe][]{PreventDefault.2013}.

Decorators sind Elemente, die nach dem Decorator-Pattern benannt sind. Zur Zeit gibt es keinerlei Unterstützung seitens der Browser zu diesem Konzept, somit wird in dieser Arbeit das vom W3C definierte Konzept nur theoretisch erläutert. Um jedoch dieses Konzept vollständig verstehen zu können, wird zuerst das genannte Pattern kurz beschrieben.

Grundsätzlich gehört das Decorator-Pattern zu den Struktur-Pattern der Softwareentwicklung. Das Pattern ist eine flexible Alternative zur Unterklassenbildung, um eine Klasse zur Laufzeit um zusätzliche Funktionalitäten erweitern zu können.

Um Decorators mit Hilfe des W3C-Konzepts näher erklären zu können, wird in diesem Kapitel folgendes Beispiel verwendet. Es gibt eine Liste von Autos, wobei jedes Auto
eine Modellbezeichnung, eine Marke, ein Bild, sowie eine Kurzbeschreibung hat. Das Markup eines Autos würde wie folgt aussehen:

caption={[Web-Components Decorators - Markup eines Autos \citereset \autocite{PreventDefault.2013}]

\begin{lstlisting}[language=HTML, caption={[Web-Components Decorators - Markup eines Autos \citereset \autocite{PreventDefault.2013}] Web-Components Decorators - Markup eines Autos}, label={lst:3_Decorators_Basic}, escapeinside={@}{@}]
<li class="car-item">
   <img class="car-image" title="Seat Ibiza" src="images/seat-ibiza.jpg" />
   <h3 class="car-model">Seat Ibiza</h3>
   <p class="car-description">The SEAT Ibiza is a supermini car manufactured by the Spanish automaker SEAT. It is SEAT&#039;s best-selling car and perhaps the most popular model in the company&#039;s range.</p>
</li>
\end{lstlisting}


Wenn die Funktionalität des Autos erweitert werden sollte, sodass es möglich ist, es sichtbar/unsichtbar zu machenbeziehungsweise schließen zu können, würde das bereits vorhandene Markup von Code-Beispiel \ref{lst:3_Decorators_Basic} auf Seite \pageref{lst:3_Decorators_Basic} wie folgt erweitert werden:

\begin{lstlisting}[language=HTML, caption={[Web-Components Decorators - Markup eines Autos mit Rahmen \citereset \autocite{PreventDefault.2013}] Web-Components Decorators - Markup eines Autos mit Rahmen}, label={lst:3_Decorators_Basic2}, escapeinside={@}{@}]
<li class="car-item">
   <section class="window-frame">
      <header>
          <a class="frame-toggle" href="#">Min/Max</a>
          <a class="frame-close" href="#">Close</a>
      </header>
      <img class="car-image" title="Seat Ibiza" src="images/seat-ibiza.jpg" />
      <h3 class="car-model">Seat Ibiza</h3>
      <p class="car-description">The SEAT Ibiza is a supermini car manufactured by the Spanish automaker SEAT. It is SEAT&#039;s best-selling car and perhaps the most popular model in the company&amp;#039;s range.</p>
   </section>
</li>
\end{lstlisting}

Code-Beispiel \ref{lst:3_Decorators_Basic2} auf Seite \pageref{lst:3_Decorators_Basic2} würde somit das Markup eines Autos und zwei Buttons beinhalten: einen zum Umschalten zwischen sichtbar und unsichtbar und einen um das Bild komplett zu löschen.
Wird das gezeigte Beispiel mit den bisherigen standardisierten Möglichkeiten umgesetzt, wird der Quellcode schnell relativ groß.

Decorators würden in diesem Beispiel bereits helfen. Es wäre möglich, spezielle Elemente im DOM mit mehr Markup, Gestaltung und zusätzlicher Funktionalität zu versehen. Essentiell hierbei ist, dass die zusätzliche Funktionalität nur für eine gewünschte Menge an Elementen erweitert werden kann. Das folgende Code-Beispiel erweitert das bereits bekannt Beispiel des Autos mit einem Decorator:

\begin{lstlisting}[language=HTML, caption={[Web-Components Decorators - Markup der zusätzlichen Funktionalität (Rahmen) \citereset \autocite{PreventDefault.2013}] Web-Components Decorators - Markup der zusätzlichen Funktionalität (Rahmen)}, label={lst:3_Decorators_Basic3}, escapeinside={@}{@}]
<decorator id="frame-decorator">
   <template>
      <section id="window-frame">
         <header>
            <a id="toggle" href="#">Min/Max</a>
            <a id="close" href="#">Close</a>
         </header>
         @\label{lst:3_Decorators_Basic3_content}@<content></content>
      </section>
   </template>
</decorator>
\end{lstlisting}

Dieses Beispiel bedarf näherer Erklärung. Decorators werden grundsätzlich mit \lstinline|<template>|-Elementen eingesetzt (mehr zu Templates in Kapitel \ref{sec:3_WC_Templates} auf Seite \pageref{sec:3_WC_Templates}). Des Weiteren wird in Zeile \ref{lst:3_Decorators_Basic3_content} des Code-Beispiels \ref{lst:3_Decorators_Basic3} ein \lstinline|<content>|-Element verwendet. Dies ist zwingend notwendig, da in dieser Stelle der Inhalt des zu dekorierenden Elements eingefügt wird. Auch ist zu erwähnen, dass in diesem Beispiel nur \lstinline|id|s verwendet werden. Dies dient der Visualisierung, dass \lstinline|id|s innerhalb eines \lstinline|<decorator>|-Elements gekapselt sind. Sie werden nie im DOM erscheinen beziehungsweise verfügbar sein. \lstinline|document.getElementById("window-frame")| wird keine Elemente zurückgeben, weder vor noch nach der Anwendung des \lstinline|<decorator>|-Elements.

Weiterhin ist es möglich, sämtliche Elemente eines Decorators zu gestalten. In Code-Beispiel \ref{lst:3_Decorators_Basic4} auf Seite \pageref{lst:3_Decorators_Basic4} werden die beiden Buttons mit \lstinline|float: right;| gestaltet. Um die \lstinline|floats| der Elemente wieder zu löschen, wird das \lstinline|<header>|-Element mit der CSS-Klasse \lstinline|clearfix| erweitert.

\begin{lstlisting}[language=HTML, caption={[Web-Components Decorators - Markup der zusätzlichen Funktionalität (Rahmen) inklusive Style \citereset \autocite{PreventDefault.2013}] Web-Components Decorators - Markup der zusätzlichen Funktionalität (Rahmen) inklusive Style}, label={lst:3_Decorators_Basic4}, escapeinside={@}{@}, escapechar=!]
<decorator id="frame-decorator">
   <template>
      <section id="window-frame">
        !\colorbox{light-gray}{<style scoped>}!
            !\colorbox{light-gray}{\#toggle {float: right;}}!
            !\colorbox{light-gray}{\#close {float: right;}}!
         !\colorbox{light-gray}{</style>}!
         !\colorbox{light-gray}{<header class="clearfix">}!
            <a id="toggle" href="#">Min/Max</a>
            <a id="close" href="#">Close</a>
         </header>
         <content></content>
      </section>
   </template>
</decorator>
\end{lstlisting}

Es ist zu beachten, dass sämtliche Gestaltungen außerhalb des \lstinline|<style scoped>|-Elements unter Verwendung der richtigen Klassennamen immer noch angewendet werden.

Um nun das bereits vorhandene Markup mit Funktionalität versehen zu können, muss zuerst noch ein wichtiger Punkt bezüglich Events erläutert werden. Bei Decorators gibt es keine \glqq normalen\grqq\ Events. Das Hinzufügen beziehungsweise Entfernen eines Decorators würde das Event, wenn es bereits auf ein Element gebunden war, löschen. Decorators bieten nun die Möglichkeit einen Event-Controller zu erstellen, um mit Hilfe von diesen Events verwalten zu können.

\begin{lstlisting}[language=HTML, caption={[Web-Components Decorators - Markup der zusätzlichen Funktionalität (Rahmen) inklusive Style und Funktionalität \citereset \autocite{PreventDefault.2013}] Web-Components Decorators - Markup der zusätzlichen Funktionalität (Rahmen) inklusive Style und Funktionalität}, label={lst:3_Decorators_Basic5}, escapeinside={@}{@}, escapechar=!]
<decorator id="frame-decorator">
   <script>
      this.listen({
         selector:"#toggle", type:"click",
         handler: function (event) {
            // do the toggle button logic here
         }
      });
      this.listen({
         selector:"#close", type:"click",
         handler: function (event) {
            // do the close button logic here
         }
      });
   </script>
   <template>
      <section id="window-frame">
         <style scoped>
            #toggle {float: right;}
            #close {float: right;}
         </style>
         <header class="clearfix">
            <a id="toggle" href="#">Min/Max</a>
            <a id="close" href="#">Close</a>
         </header>
         <content></content>
      </section>
   </template>
</decorator>
\end{lstlisting}


Der in Code-Beispiel \ref{lst:3_Decorators_Basic5} auf Seite \ref{lst:3_Decorators_Basic5} gezeigte Decorator kann somit verwendet werden, um die gewünschte Funktionalität (das Bild des Autos soll sichtbar, unsichtbar beziehungsweise gelöscht werden können) bereitstellen zu können, ohne dabei für jedes \lstinline|<li>|-Element extra Markup hinzufügen zu müssen. Schlussendlich würde das Beispiel für ein Auto wie folgt aussehen:

\begin{lstlisting}[language=HTML, caption={[Web-Components Decorators - Markup eines Autos mit Decorator \citereset \autocite{PreventDefault.2013}] Web-Components Decorators - Markup eines Autos mit Decorator}, label={lst:3_Decorators_Basic6}, escapeinside={@}{@}, escapechar=!]
<decorator id="frame-decorator">
   <script>
      this.listen({
         selector:"#toggle", type:"click",
         handler: function (event) {
            // do the toggle button logic here
         }
      });
      this.listen({
         selector:"#close", type:"click",
         handler: function (event) {
            // do the close button logic here
         }
      });
   </script>
   <template>
      <section id="window-frame">
         <style scoped>
            #toggle {float: right;}
            #close {float: right;}
         </style>
         <header class="clearfix">
            <a id="toggle" href="#">Min/Max</a>
            <a id="close" href="#">Close</a>
         </header>
         <content></content>
      </section>
   </template>
</decorator>

<li class="car-item">
   <img class="car-image" title="Seat Ibiza" src="images/seat-ibiza.jpg" />
   <h3 class="car-model">Seat Ibiza</h3>
   <p class="car-description">The SEAT Ibiza is a supermini car manufactured by the Spanish automaker SEAT. It is SEAT&#039;s best-selling car and perhaps the most popular model in the company&#039;s range.</p>
</li>
\end{lstlisting}

\begin{lstlisting}[language=CSS, caption={[Web-Components Decorators - CSS für die Verwendung von Decorators \citereset \autocite{PreventDefault.2013}] Web-Components Decorators - CSS für die Verwendung von Decorators}, label={lst:3_Decorators_Basic7}, escapeinside={@}{@}, escapechar=!]
.car-item {
  decorator: url(#frame-decorator);
}
\end{lstlisting}

Unter der Verwendung des in Code-Beispiel \ref{lst:3_Decorators_Basic7} auf Seite \pageref{lst:3_Decorators_Basic7} gezeigten CSS-Attributs wird der Decorator für das in Code-Beispiel \ref{lst:3_Decorators_Basic6} auf Seite \pageref{lst:3_Decorators_Basic6} gezeigte Markup verwendet.


\subsubsection{Custom Elements}
\label{sec:3_WC_Elements}

Custom Elemente sind ein neue Typen von bisher bestehenden DOM-Elementen. Sie können vom Autor beliebig definiert werden und müssen nur wenige Vorschriften einhalten. Im Gegensatz zu Decorators (siehe Kapitel \ref{sec:3_WC_Decorators} auf Seite \pageref{sec:3_WC_Decorators}), welche zustandslos und kurzlebig sind, können Custom Elements den Zustand kapseln und eine Schnittstelle zur Verwendung bereitstellen. Tabelle \ref{tab:Unterschiede} zeigt die Schlüsselunterschiede zwischen den beiden Konzepten.

\begin{table}[h]
\centering
\begin{tabular}{ l || l | l}
& Decorators & Custom Elements \\
\hline
\hline
Lebensdauer & kurzlebig, wenn ein passender CSS-Selector vorhanden ist & stabil, angepasst an die Lebensdauer des Elements\\
\hline
dynamisches hinzufügen, entfernen & Ja, auf Basis des CSS-Selectors & Nein; einmalig (bei der Erstellung des Elements)\\
\hline
In einem Skript erreichbar& Nein; transparent zum DOM und kein Hinzufügen einer Schnittstelle möglich & Ja, im DOM erreichbar und eventuell vorhandene Schnittstelle\\
\hline
Zustand & zustandsloser Ansatz & zustandsorientiertes DOM-Objekt \\
\hline
Behaviour & Simulated by changing decorators & First-class using script and events \\
\end{tabular}
\caption[
Schlüsselunterschiede zwischen Decorators und Custom Elements
]
{Schlüsselunterschiede zwischen Decorators und Custom Elements}
\label{tab:Unterschiede}
\end{table}

\subsection{Shadow DOM}
\label{sec:3_WC_Shadow_DOM}

Das folgende Kapitel basiert ausschließlich auf der Spezifikation von Shadow-DOM des W3C \citereset \autocite[siehe][]{GlazkovShadowDOM.2013} und auf dem Artikel \glqq Shadow DOM 101\grqq\ \citereset \autocite[siehe][]{Cooney.2013}.

Mit Hilfe von Shadow-DOM können Elemente mit einer neuen Art von Knoten verbunden werden. Diese neue Art von Knoten wird auch \glqq Shadow-Root\grqq\ genannt. Ein Element, dass einer Shadow-Root zugeordnet ist, wird auch \glqq Shadow-Host\grqq\ bezeichnet. Anstatt den Inhalt eines Shadow-Hosts zu rendern, wird immer der des Shadow-Roots gerendert.

\begin{lstlisting}[language=HTML, caption={[Shadow-Root Beispiel eines Buttons \citereset \autocite{Cooney.2013}] Shadow-Root Beispiel eines Buttons}, label={lst:3_ShadowDomBasic1}, escapeinside={@}{@}]
<button>Hello, world!</button>
<script>
  var host = document.querySelector('button');
  var root = host.createShadowRoot();
  root.textContent = 'Hello, shadow DOM!';
</script>
\end{lstlisting}

Code-Beispiel \ref{lst:3_ShadowDomBasic1} rendert zuerst das in Abbildung \ref{sfig:3_ShadowDom1} auf Seite \pageref{sfig:3_ShadowDom1} gezeigte Ergebnis. Danach wird mit Hilfe von JavaScript und Shadow-DOM das Element, wie in Abbildung \ref{sfig:3_ShadowDom2} auf Seite \pageref{sfig:3_ShadowDom2} zu sehen ist, verändert.

\begin{figure}[h]
  \centering
  \subfloat[HTML gerendertes Element]{
    \includegraphics[]{images/SS2.png}
    \label{sfig:3_ShadowDom1}
  }
  \qquad
  \subfloat[HTML Element mit Hilfe von JavaScript und Shadow DOM manipuliert]{
    \includegraphics[]{images/SS1.png}
    \label{sfig:3_ShadowDom2}
  }
  \caption[
    Beispiel einer Shadow-Root Node
  ]{
    Beispiel einer Shadow-Root Node
  }
  \label{sfig:3_ShadowDom}
\end{figure}

Wenn das \lstinline|<button>|-Element nach seinem Inhalt mittels der \lstinline|textContent|-Eigenschaft abgefragt wird, wird das Resultat nicht \lstinline|"Hello, shadow DOM!| zurückgeben, sondern \lstinline|"Hello, world!"|, da die DOM-Unterstruktur unter der Shadow-Root vollständig gekapselt ist.

Es ist zu erwähnen, dass dies ein sehr schlechtes Beispiel für Suchmaschinen, Browser-Extension, Screen-Readers etc. ist, da sämtlicher Inhalt des Shadow-DOMs nicht für diese erreichbar ist. Shadow-DOM ist nur für sematisch bedeutungsloses Markup, das benötigt wird um eine Webkomponente zu erstellen, gedacht.

\subsubsection{Trennung von Inhalt und Darstellung}
\label{sec:3_WC_Shadow_DOM1}

Code-Beispiel \ref{lst:3_ShadowDomBasic2} auf Seite \pageref{lst:3_ShadowDomBasic2} wird als Ausgangsbasis dieses Beispiels genommen \citereset \autocite[siehe][]{Cooney.2013}. Abbildung \ref{fig:3_ShadowDom2} auf Seite \pageref{fig:3_ShadowDom2} zeigt diese Grundbasis, um mit darauffolgenden Schritten sämtlichen Inhalt von der Darstellung zu trennen. Dies garantiert, dass der tatsächliche Inhalt für Suchmaschinen, Screen-Readers, Browser-Extensions etc. erreichbar und die Darstellung für die Endbenutzerin beziehungsweise den Endbenutzer \glqq unsichtbar\grqq\ ist.

\begin{lstlisting}[language=HTML, caption={Namensschild ohne Shadow-DOM - Ausgangsbasis um Inhalt von Darstellung zu trennen}, label={lst:3_ShadowDomBasic2}, escapeinside={@}{@}]
<style>
.outer {
  border: 2px solid brown;
  border-radius: 1em;
  background: red;
  font-size: 20pt;
  width: 12em;
  height: 7em;
  text-align: center;
}
.boilerplate {
  color: white;
  font-family: sans-serif;
  padding: 0.5em;
}
.name {
  color: black;
  background: white;
  font-family: "Marker Felt", cursive;
  font-size: 45pt;
  padding-top: 0.2em;
}
</style>
<div class="outer">
  <div class="boilerplate">
    Hi! My name is
  </div>
  <div class="name">
    Bob
  </div>
</div>
\end{lstlisting}

\begin{figure}[h]
\centering
\includegraphics[height=5.0cm]{images/SS3.png}
\caption[
  Ausgangsbeispiel von der Trennung von Darstellung und Inhalt bei Shadow-DOM \citereset \autocite{Cooney.2013}
]{Ausgangsbeispiel von der Trennung von Darstellung und Inhalt bei Shadow-DOM}
\label{fig:3_ShadowDom2}
\end{figure}

Dadurch, dass dem DOM-Tree Kapselung fehlt, ist die gesamte Struktur des Namensschildes im Dokument sichtbar. Wenn beispielsweise externe Elemente auf der Webseite dieselben Klassennamen verwenden, würden diverse CSS-Klassen überschrieben werden.

\begin{enumerate}
\litem{Verstecken von Darstellungsdetails} \hfill \\
Semantisch gibt es nur zwei wichtige Informationen bei diesem Beispiel:
\begin{enumerate}
\item Es ist ein Namensschild
\item Der Name ist \glqq Bob\grqq .
\end{enumerate}
Daraus wird das Markup erstellt, das semantisch näher bei der gewünschten Information ist (siehe Code-Beispiel \ref{lst:3_ShadowDomBasic3}).

\begin{lstlisting}[language=HTML, caption={[Darstellung des Markups mit der gewünschten Information ohne Darstellung \citereset \autocite{Cooney.2013}] Darstellung des Markups mit der gewünschten Information ohne Darstellung}, label={lst:3_ShadowDomBasic3}, escapeinside={@}{@}]
<div id="nameTag">Bob</div>
\end{lstlisting}

Des Weiteren wird sämtlicher Code, der der Darstellung dient, in ein \lstinline|<template>|-Element gepackt (siehe Code-Beispiel \ref{lst:3_ShadowDomBasic4}). Dies ist notwendig um sämtliche Darstellungsdetails von dem eigentlichen Inhalt trennen zu können.

\begin{lstlisting}[language=HTML, caption={[Darstellung des Markups mit der gewünschten Information mit Hilfe von einem Template \citereset \autocite{Cooney.2013}] Darstellung des Markups mit der gewünschten Information mit Hilfe von einem Template}, label={lst:3_ShadowDomBasic4}, escapeinside={@}{@}]
<div id="nameTag">Bob</div>
<template id="nameTagTemplate">
  <style>
    .outer {
      border: 2px solid brown;
      border-radius: 1em;
      background: red;
      font-size: 20pt;
      width: 12em;
      height: 7em;
      text-align: center;
    }
    .boilerplate {
      color: white;
      font-family: sans-serif;
      padding: 0.5em;
    }
    .name {
      color: black;
      background: white;
      font-family: "Marker Felt", cursive;
      font-size: 45pt;
      padding-top: 0.2em;
    }
  </style>
  <div class="outer">
    <div class="boilerplate">
      Hi! My name is
    </div>
    <div class="name">
      Bob
    </div>
  </div>
</template>
\end{lstlisting}

Zu diesem Zeitpunkt wird \glqq Bob\grqq\ das einzige sein, das gerendert wird. Das Template enthält sämtlichen Code der Darstellung und muss nun beispielsweise mit JavaScript hinzugefügt werden. In Code-Beispiel \ref{lst:3_ShadowDomBasic5} auf Seite \pageref{lst:3_ShadowDomBasic5} wird zuerst eine Shadow-Root am Element \lstinline|<div id=nameTag></div>| erstellt. Danach wird nach dem Template gesucht und der Inhalt dieses Templates an die Shadow-Root angefügt.

\begin{lstlisting}[language=JavaScript, caption={[Hinzufügen des Inhalts eines Templates in eine Shadow-Root \citereset \autocite{Cooney.2013}] Hinzufügen des Inhalts eines Templates in eine Shadow-Root}, label={lst:3_ShadowDomBasic5}, escapeinside={@}{@}]
<script>
  var shadow = document.querySelector('#nameTag').createShadowRoot();
  var template = document.querySelector('#nameTagTemplate');
  shadow.appendChild(template.content);
</script>
\end{lstlisting}

Da nun eine Shadow-Root mit Markup vorhanden ist, wird das Namensschild wieder korrekt angezeigt. Wenn das Element mit Hilfe der Entwickler-Tools im Browser inspiziert wird, wird nur die gewünschte Information ohne Darstellungselemente angezeigt. Dies zeigt, dass durch die Verwendung von Shadow-DOM sämtliche Darstellungendetails im Shadow-DOM gekapselt wurden und von außen nicht erreichbar sind.

\litem{Trennung von Inhalt und Darstellung} \hfill \\
Mit Hilfe von Code-Beispiel \ref{lst:3_ShadowDomBasic4} und \ref{lst:3_ShadowDomBasic5} werden sämtliche Darstellungsdetails versteckt, jedoch wurde der Inhalt noch nicht mit der Darstellung getrennt. Wenn beispielsweise der Name des Namensschildes ausgetauscht werden müsste, müsste dies an zwei Stellen gemacht werden: Einerseits an der Stelle im Template und andererseits an der Stelle des \lstinline|<div id=nameTag></div>|-Elements.

Um den tatsächlichen Inhalt von sämtlichen Darstellungen zu trennen, muss eine Komposition von Elementen benutzt werden. Das Namensschild setzt sich einerseits aus dem roten Hintergrund mit den \glqq Hi! My name is\grqq -Text zusammen und andererseits aus dem Namen der Person.

Als Programmiererin beziehungsweise Programmierer einer Komponente kann entschieden werden, wie die Komposition des erstellten Elements funktionieren soll. Mit Hilfe des \lstinline|<content>|-Elements können Kompositionen erstellt werden. Dieses Element erstellt einen \glqq Insertion-Point\grqq\ in der Darstellung und sucht Inhalte aus dem Shadow-Host, die an dieser Stelle angezeigt werden sollten.

\begin{lstlisting}[language=HTML, caption={[Erweiterung des Code-Beispiels \ref{lst:3_ShadowDomBasic4} mit dem <content>-Element \citereset \autocite{Cooney.2013}] Erweiterung des Code-Beispiels \ref{lst:3_ShadowDomBasic4} mit dem <content>-Element}, label={lst:3_ShadowDomBasic6}, escapeinside={@}{@}]
<div id="nameTag">Bob</div>
<template id="nameTagTemplate">
  <style>
    .outer {
      border: 2px solid brown;
      border-radius: 1em;
      background: red;
      font-size: 20pt;
      width: 12em;
      height: 7em;
      text-align: center;
    }
    .boilerplate {
      color: white;
      font-family: sans-serif;
      padding: 0.5em;
    }
    .name {
      color: black;
      background: white;
      font-family: "Marker Felt", cursive;
      font-size: 45pt;
      padding-top: 0.2em;
    }
  </style>
  <div class="outer">
    <div class="boilerplate">
      Hi! My name is
    </div>
    <div class="name">
      <content></content>
    </div>
  </div>
</template>
\end{lstlisting}

In Code-Beispiel \ref{lst:3_ShadowDomBasic6} auf Seite \pageref{lst:3_ShadowDomBasic6} wird das Namensschild mit dem vom Shadow-Host projizierten Inhalt in das \lstinline|<content>|-Element gerendert. Dies vereinfacht die Struktur des Dokuments, da der Name nur noch an einer Stelle vorhanden ist. Müsste nun der Name aktualisiert werden, wäre das mit folgender Methode möglich: \lstinline|document.querySelector('#nameTag').textContent = 'Shellie';|.

Das Namensschild wird automatisch nach Zuweisung eines neuen Namens aktualisiert, da der Inhalt vom Namensschild in das \lstinline|<content>|-Element projiziert wird. Somit wurde die Trennung von Inhalt und Darstellung erreicht.
\end{enumerate}

\subsubsection{HTML Imports}
\label{sec:3_WC_Imports}

Es gibt eine Vielzahl von Möglichkeiten, wie man diverse Ressourcen lädt. Für JavaScript gibt es beispielsweise den \lstinline|<script src=''>|-Tag, für CSS gibt es den \lstinline|<link rel='stylesheet'>|-Tag. Weiters gibt es eigene Tags für Bilder, Video, Audio, etc. Die meisten Web-Inhalte haben einen einfachen, deklarativen Weg sie zu laden. HTML hingegen besitzt keinen standardisierten Weg. Zurzeit gibt es folgende Weg um HTML zu laden:
\begin{enumerate}
\item \lstinline|<iframe>|
\item AJAX
\item \lstinline|<script type='text/html'>|
\end{enumerate}

Jede dieser Methoden bringt seine Vor- und Nachteile mit sich und keine dieser Methoden ist eine standardisierte Weise, wie man externes HTML lädt.

HTML-Imports bieten einen standardisierten Weg, wie man ein HTML-Dokument in ein anderes HTML-Dokument lädt. Ein HTML-Import ist des Weiteren nicht auf Markup limitiert, sondern es kann auch CSS, JavaScript, etc. beinhalten. Code-Beispiel \ref{lst:3_Imports_Basic1} auf Seite \pageref{lst:3_Imports_Basic1} zeigt, wie man ein lokales HTML-Dokument lädt.

\begin{lstlisting}[language=HTML, caption={Laden eines lokalen HTML-Dokuments}, label={lst:3_Imports_Basic1}, escapeinside={@}{@}]
<head>
  <link rel="import" href="path/to/imports/car.html"
</head>
\end{lstlisting}

Die URL eines Imports nennt man auch \glqq Import-Stelle\grqq . Um Inhalt von einer anderen Domain zu laden, muss die Import-Stelle CORS\footnote{Cross-Origin Resource Sharing} aktiviert sein. Code-Beispiel \ref{lst:3_Imports_Basic2} auf Seite \pageref{lst:3_Imports_Basic2} zeigt, wie man ein externes HTML-Dokument lädt.

\begin{lstlisting}[language=HTML, caption={Laden eines externen HTML-Dokuments}, label={lst:3_Imports_Basic2}, escapeinside={@}{@}]
<head>
  <!-- Resources on other origins must be CORS-enabled. -->
  <link rel="import" href="http://example.com/car.html">
</head>
\end{lstlisting}

Der Netzwerk-Stack des Browsers entfernt sämtliche Duplikate bezüglich Requests von derselben URL. Dies bedeutet, dass bei Importe, die dieselbe URL haben, nur einmal aufgerufen werden.

\textbf{Ressourcen bündeln}

Importe stellen Konventionen bereit um HTML, CSS, JavaScript etc. bündeln zu können, damit sie als eine Datei lieferbar sind. Dies ist eine sehr wesentliche und mächtige Eigenschaft von HTML-Importe. Durch die Funktionen, die HTML-Importe bereitstellen, ist es möglich, eine Web-Applikation in einzelne, logische Segmente aufzuteilen und dem Endbenutzer dennoch nur eine URL geben zu müssen.

Ein sehr gutes Beispiel für einen sinnvollen Einsatz von HTML-Importe wäre Bootstrap\footnote{Mehr Information zu Dojo Toolkit unter \href{http://getbootstrap.com/}{http://getbootstrap.com/}}. Bootstrap besteht aus individuellen CSS- und JavaScript-Dateien, sowie Fonts. Darüber hinaus benötigt es für die bereitgestellten Plugins JQuery. Code-Beispiel \ref{lst:3_Imports_Basic3} auf Seite \pageref{lst:3_Imports_Basic3} zeigt, wie man Bootstrap auf mehrere Dokumente aufteilen und laden könnte.

\begin{lstlisting}[language=HTML, caption={Inkludierung von Bootstrap mit Hilfe von einem HTML-Import}, label={lst:3_Imports_Basic3}, escapeinside={@}{@}]
<!-- main.html -->
<head>
  <link rel="import" href="bootstrap.html">
</head>

<!-- bootstrap.html -->
<link rel="stylesheet" href="bootstrap.css">
<link rel="stylesheet" href="fonts.css">
<script src="jquery.js"></script>
<script src="bootstrap.js"></script>
<script src="bootstrap-tooltip.js"></script>
<script src="bootstrap-dropdown.js"></script>
...
\end{lstlisting}

\textbf{Load/Error Event-Handling}

Das \lstinline|<link>|-Element feuert ein \glqq Load\grqq -Event, wenn ein HTML-Import erfolgreich geladen wurde und ein \glqq Error\grqq -Event, wenn dies nicht der Fall ist. Code-Beispiel \ref{lst:3_Imports_Basic4} auf Seite \pageref{lst:3_Imports_Basic4} zeigt ein Beispiel von Error-Handling bei HTML-Importe.

\begin{lstlisting}[language=HTML, caption={Error-Handling bei HTML-Importe}, label={lst:3_Imports_Basic4}, escapeinside={@}{@}]
<head>
  <script>
  function handleLoad(e) {
    console.log('Loaded import: ' + e.target.href);
  }
  function handleError(e) {
    console.log('Error loading import: ' + e.target.href);
  }
</script>

  <link rel="import" href="car.html" onload="handleLoad(event)" onerror="handleError(event)">
</head>
\end{lstlisting}

Ein wichtiger Punkt bei dem in Code-Beispiel \ref{lst:3_Imports_Basic4} gezeigten Beispiel ist, dass die Funktionen vor dem Import definiert wurden. Der Browser versucht einen HTML-Import dann zu laden, wenn er dem Tag begegnet. Wenn zu diesem Zeitpunkt die Funktionen noch nicht existieren, würden Fehler in der Konsole ausgegeben werden, da die Funktionsnamen noch \lstinline|undefined| sind.

\textbf{Benutzung des Inhalts eines Imports}

Wenn man einen HTML-Import benutzt, bedeutet dies nicht, dass an der Stelle, wo der Import-Befehl geschrieben  wird, der Inhalt des Imports platziert wird. Vielmehr bedeutet es, dass der Browser das zu importierende Dokument analysiert und es lädt, um es dann für weitere Verwendung bereit stellen zu können. Wenn man den Inhalt eines Imports erreichen will, da es beispielsweise gewisse \lstinline|<template>|-Elemente beinhaltet, muss man dafür JavaScript verwenden. Code-Beispiel \ref{lst:3_Imports_Basic5} auf Seite \pageref{lst:3_Imports_Basic5} holt sich den Inhalt eines HTML-Imports. Die importierte Datei (warnings.html) beinhaltet diverses gestaltete Markup, was in der Hauptseite (main.html) verwendet werden sollte. Des Weiteren wird nur ein spezieller Teil des Imports verwendet, nämlich das \lstinline|<div>|-Element mit der Klasse \lstinline|warning|. Der restliche Inhalt des importierten HTML-Dokuments bleibt inaktiv und wird nicht vom Browser gerendert.

\begin{lstlisting}[language=JavaScript, caption={Klonen des Inhalts eines HTML-Imports}, label={lst:3_Imports_Basic5}, escapeinside={@}{@}]
<!-- warnings.html -->
<div class="warning">
  <style scoped>
    h3 {
      color: red;
    }
  </style>
  <h3>Warning!</h3>
  <p>This page is under construction</p>
</div>

<div class="outdated">
  <h3>Heads up!</h3>
  <p>This content may be out of date</p>
</div>


<!-- main.html -->
<head>
  <link rel="import" href="warnings.html">
</head>
<body>
    <script>
    var link = document.querySelector('link[rel="import"]');
    var content = link.import;

    // Grab specific DOM from warning.html's document.
    var el = content.querySelector('.warning');

    document.body.appendChild(el.cloneNode(true));
  </script>
</body>
\end{lstlisting}

\textbf{JavaScript in einem zu importierendem Dokument}

Importe befinden sich nicht im Hauptdokument. Sie können als Satelliten zum Hauptdokument gesehen werden. Im zu importierenden Dokument kann der DOM vom Hauptdokument und sein eigenes DOM erreicht werden. Code-Beispiel \ref{lst:3_Imports_Basic6} auf Seite \pageref{lst:3_Imports_Basic6} zeigt, wie das zu importierende Dokument eines seiner Stylesheets selbst im Hauptdokument hinzufügt. Wichtig hierbei ist, dass das zu importierende Dokument einerseits eine Referenz zum eigenen Dokument und andererseits eine Referenz zum Hauptdokument beinhaltet.

\begin{lstlisting}[language=JavaScript, caption={JavaScript im HTML-Import, um Inhalt automatisch im Hauptdokument hinzuzufügen}, label={lst:3_Imports_Basic6}, escapeinside={@}{@}]
<link rel="stylesheet" href="http://www.example.com/styles.css">
<link rel="stylesheet" href="http://www.example.com/styles2.css">
<script>
  // importDoc references this import's document
  var importDoc = document.currentScript.ownerDocument;

  // mainDoc references the main document (the page that's importing us)
  var mainDoc = document;

  // Grab the first stylesheet from this import, clone it,
  // and append it to the importing document.
  var styles = importDoc.querySelector('link[rel="stylesheet"]');
  mainDoc.head.appendChild(styles.cloneNode(true));
</script>
\end{lstlisting}

Ein Skript in einem Import kann entweder Code direkt ausführen, oder Funktionen bereitstellen, die dem importierenden Dokument zur Verfügung stehen. Grundregeln für JavaScript in einem HTML-Import sind folgende:
\begin{itemize}
\item Skripte im Import werden im Kontext des importierenden Dokuments aufgerufen. Das bedeutet, dass \lstinline|window.document| im Import-Dokument eine Referenz zum Dokument ist, dass die Datei importiert. Dies hat zur Folge, dass sämtliche Funktionen, die im Import-Dokument definiert werden zum \lstinline|window|-Objekt hinzugefügt werden. Weiters ist es nicht erforderlich \lstinline|<script>|-Blöcke im Hauptdokument hinzuzufügen, da sie ausgeführt werden.
\item Importe blocken den Browser nicht beim Parsen des Hauptdokuments. Dennoch werden Skripte in den Importen der Reihe nach verarbeitet.
\end{itemize}

\textbf{HTML-Importe im Zusammenhang mit Templates}

Ein sehr großer Vorteil, wenn diese beiden Unterpunkte von Web-Components zusammen verwendet werden ist, dass Skripte innerhalb eines Templates nicht beim Laden des zu importierenden Dokuments ausgeführt werden, sondern erst dann, wenn das Template aktiv wird, sprich dem DOM des Hauptdokuments hinzugefügt wird.

\textbf{HTML-Importe im Zusammenhang mit benutzerdefinierten Elementen}

Wenn man diese beiden Technologien vereint, muss sich der Benutzer, der beispielsweise ein fremdes, benutzerdefiniertes Element mit Hilfe eines HTML-Imports lädt, nicht um die Registrierung des Elements kümmern, da es bereits im zu importierenden Dokument gemacht werden kann.

\textbf{Abhängigkeits-Management und Sub-Importe}

Sub-Importe sind vor allem dann vom Vorteil, wenn eine Komponente wiederverwendbar oder erweitert werden soll. Beispielsweise kann man JQuery als eine Komponente ansehen und sie als HTML-Import definieren. Wenn man mehrere, benutzerdefinierte Elemente mit Hilfe von JQuery entwickelt und sie anderen zur Verfügung stellt, werden die Abhängigkeiten für sämtliche Elemente automatisch geladen. Nimmt man an, das man drei benutzerdefinierte Elemente lädt, wobei jedes einzelne Element an sich JQuery als Abhängigkeit mittels HTML-Import geladen hat, wird JQuery trotzdem nur ein einziges Mal im Hauptdokument geladen.

\subsubsection{Browser Unterstützung}
\label{sec:3_WC_Support}

Die nachstehende Tabelle \ref{tab:BrowserSupport} zeigt die Unterstützung der Browser bezüglich Web-Components. Sie wurde zuletzt am 13 März 2014 aktualisiert und ist somit aktuell. Sämtliche Browser aus der Tabelle besitzen die aktuellste Version. Grün bedeutet, dass die Technologie in dem jeweiligen Browser als stabil gewertet wird und verwendet werden kann. Gelb bedeutet, dass es vom jeweiligen Browser in Arbeit ist, noch Bugs auftreten, oder mittels einer Flag erreichbar ist. Rot bedeutet, dass es keine Information bezüglich der Technologie in dem jeweiligen Browser gibt.

\begin{description}
\litem{Chrome} \hfill \\
Was Web-Components betrifft ist Googles Chrome der goldene Standard. Sie haben sich an die Spitze gesetzt, was die Umsetzung der Spezifikation angeht. Drei Technologien von Web-Components werden bereits als stabil in Chrome bezeichnet. Auch HTML-Importe sind bereits vorhanden, jedoch um diese benutzen zu können, ist es erforderlich, eine Flag im Browser zu aktivieren.

\litem{Opera} \hfill \\
Dadurch, dass Opera seit einiger Zeit auf die Basus von Chromium (Blink) gewechselt hat, wird von Opera der gleiche Weg wie von Google erwartet. Zur Zeit gibt es keine kennbaren Unterschiede was die Implementierung der Spezifikation angeht bezüglich den beiden Browsern.

\litem{Firefox} \hfill \\
Auch Mozilla versucht mit Firefox den Standard schnellstmöglich umzusetzen, jedoch gibt es einige Bugs diesbezüglich. Nur Templates werden bis jetzt als stabil angesehen und Custom-Elements, sowie Shadow-DOM sind nur mittels einer Flag erreichbar.

\litem{Safari} \hfill \\
Obwohl viele Funktionen von Web-Components in Webkit implementiert wurden, wurden sie nie in Safari verwendet beziehungsweise zur Verfügung gestellt. Mit der Abzweigung des Chromium-Port von Safari wurde begonnen sämtliche Web-Components Funktionen aus dem Browser zu entfernen.

\litem{Internet Explorer} \hfill \\
Microsoft gibt kein öffentliches Statement bezüglich ihren Entwicklungsplänen ab und somit ist es nicht klar, inwiefern sie die Technologien von Web-Components implementieren werden. Der vor kurzem veröffentlichte Internet Explorer 11 scheint keine der Schnittstellen für Web-Components zu beinhalten.

\end{description}

\begin{table}[h]
\begin{tabular}{ M{2cm} || M{2cm} | M{2cm} | M{2cm} | M{2cm} | M{2cm} N}
& Chrome & Opera & Firefox & Safari & IE \\
\hline
\hline
Templates & \cRect{green}{1cm}{1cm} & \cRect{green}{1cm}{1cm} & \cRect{green}{1cm}{1cm} & \cRect{red}{1cm}{1cm} & \cRect{red}{1cm}{1cm} &\\[8ex] \hline
HTML-Importe & \cRect{yellow}{1cm}{1cm} & \cRect{yellow}{1cm}{1cm} & \cRect{yellow}{1cm}{1cm} & \cRect{red}{1cm}{1cm} & \cRect{red}{1cm}{1cm} &\\[8ex] \hline
Custom Elements & \cRect{yellow}{1cm}{1cm} & \cRect{yellow}{1cm}{1cm} & \cRect{yellow}{1cm}{1cm} & \cRect{red}{1cm}{1cm} & \cRect{red}{1cm}{1cm} &\\[8ex] \hline
Shadow DOM & \cRect{green}{1cm}{1cm} & \cRect{green}{1cm}{1cm} & \cRect{yellow}{1cm}{1cm} & \cRect{red}{1cm}{1cm} & \cRect{red}{1cm}{1cm} &\\[8ex]
\end{tabular}
\caption[
Browser Unterstützung von Web-Components (Stand 13.03.2014)
\small\texttt{http://jonrimmer.github.io/are-we-componentized-yet/}
]
{Browser Unterstützung von Web-Components}
\label{tab:BrowserSupport}
\end{table}

Listing \ref{lst:3_Browser_Support} auf Seite \pageref{lst:3_Browser_Support} zeigt, wie man die Funktionen beziehungsweise einzelnen Technologien auf Verfügbarkeit im Browser testen kann. Das Ergebnis kann in der Konsole des benutzten Browsers eingesehen werden.

\begin{lstlisting}[language=JavaScript, caption={Feature-Detection für Web-Components}, label={lst:3_Browser_Support}, escapeinside={@}{@}]
function supportsTemplate() {
  return 'content' in document.createElement('template');
}
function supportsCustomElements() {
  return 'registerElement' in document;
}
function supportsImports() {
  return 'import' in document.createElement('link');
}
function supportsShadowDom(){
  return typeof document.createElement('div').createShadowRoot === 'function';
}

(function() {
  supportsTemplate()? console.log("Templates are supported!") : console.error("Templates are not supported!")
  supportsCustomElements()? console.log("Custom elements are supported!") : console.error("Custom elements are not supported!")
  supportsImports()? console.log("HTML-Imports are supported!") : console.error("HTML-Imports are not supported!")
  supportsShadowDom()? console.log("Shadow-DOMs are supported!") : console.error("Shadow-DOMs are not supported!")
}());
\end{lstlisting}

\subsection{Google Polymer}
\label{sec:3_Polymer}


In allgemeinen Worten abgefasst ist Google-Polymer ein Framework, das sämtliche Funktionen von Web-Components zur Verfügung stellt. Polymer verfolgt den Ansatz, dass alles was entwickelt wird, ist eine Web-Komponente und diese entwickelt sich mit dem Web mit. Abbildung \ref{fig:3_polymer_architecture} auf Seite \pageref{fig:3_polymer_architecture} zeigt die Architektur von Polymer.


\begin{figure}[h]
\centering
\includegraphics[height=5.0cm]{images/polymer_architecture.png}
\caption[
Polymers Architektur, Urldate: 04.2014 \newline
\small\texttt{http://i.stack.imgur.com/Ksn6s.png}
]{Polymers Architektur}
\label{fig:3_polymer_architecture}
\end{figure}

\begin{description}
\item[Die rote Schicht] visualisiert die Polyfills, die Polymer zur Verfügung stellt. Diese erlauben die Benutzung von Web-Components. Wichtig hierbei ist, dass die Größe dieser Polyfill-Bibliotheken mit der Weiterentwicklung der Browser abnimmt. Dies bedeutet, dass je mehr Funktionalität von der Spezifikation in den Browsern implementiert ist, desto kleiner sind die Polyfill-Bibliotheken. Der Idealfall für Polymer wäre, dass sämtliche Zusatzbibliotheken, die die nativen Browser-Funktionen emulieren, nicht mehr gebraucht werden.
\item[Die gelbe Schicht] stellt die Meinung von Google dar, wie die spezifizierten Browser Schnittstellen zu Web-Components zusammen verwendet werden sollen. Zusätzlich zu den spezifizierten Technologien werden des Weiteren Fuktionalitäten wie \glqq data-bindings\grqq , \glqq change watcher\grqq , \glqq öffentliche Eigenschaften\grqq , etc.
\item[Die grüne Schicht] repräsentiert eine umfassende Reihe von Interface-Komponenten. Diese entwickelt sich ständig weiter und basieren auf der gelben, sowie roten Schicht.
\end{description}

Polymer bietet die Möglichkeit neben der Erstellung benutzerdefinierter Elemente auch die Verwendung von vordefinierten Elementen. Ein Beispiel für ein vordefiniertes Element wäre das \lstinline|<polymer-ajax>|-Element. Es erscheint in erster Linie als nicht sehr nützlich, jedoch versucht es, einen Standard für Entwickler bereitzustellen, um Ajax-requests zu erstellen beziehungsweise abzuwickeln. Dieses Element ist ähnlich zu folgender Funktion: \lstinline{$.ajax()}\footnote{Mehr Information zur jQuery.ajax-Funktion unter \href{http://api.jquery.com/jQuery.ajax/}{http://api.jquery.com/jQuery.ajax/}}. Der Unterschied zwischen den beiden Möglichkeiten, einen Ajax-Request abzuwickeln, ist, dass die \lstinline{$.ajax()}-Methode Abhängigkeiten besitzt, wohingegen die \lstinline|<polymer-ajax>|-Methode vollkommen unabhängig ist.

Wie man benutzerdefinierte Elemente mittels Polymer entwickelt und welche Coding-Richtlienien dieses Framework beinhaltet, wird in Kapitel \ref{sec:4_WC_Polymer} auf Seite \pageref{sec:4_WC_Polymer} genauer erläutert.

\subsection{Konklusio}
\label{sec:3_Konklusion}

\todo[inline]{Hauptfunktion von Web-Components nochmal erläutern}
\todo[inline]{Vorteile die dadurch entstehen}
\todo[inline]{Nachteile (bei Shadow-dom seo friendly)}
\todo[inline]{Browser-Unterstützung}
\todo[inline]{erster Bezug von der Erstellung von Komponenten zu komponenbenbasierter Softwareentwicklung/Softwarearchitektur}



