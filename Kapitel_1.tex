\section{Einführung}
\label{sec:1_Einführung}

\todo[inline]{Remove Color of Hyperref package (see Bakk.tex comments).}
\todo[inline]{Check version of Polymer}
\todo[inline]{Check version of W3C draft}
\todo[inline]{Update and rewrite Chapter 1 completly}
\todo[inline]{Put connections between paragraphs in Chapter 2}
\todo[inline]{Referencing original papers/books with: autocite{X} nach autocite{Y}}
\todo[inline]{Look up if there is any component collection available}
\todo[inline]{Nutzwerkanalyse von Web-Components im Allgemeinen}
\todo[inline]{Nutzwerkanalyse von Web-Components im allgemeinen}


\subsection{Relevanz}
\label{sec:1_Relevanz}
\todo{Wird zum Schluss geschrieben}
%Taken from golem.de
%Web-Components gehören zu den interessantesten neuen Webtechniken, denn sie haben das Potenzial, die Entwicklung von Web-Apps enorm zu vereinfachen und zu beschleunigen. Damit kann jeder seine eigenen, komplexen HTML-Elemente selbst bauen oder von anderen erschaffene Elemente in der eigenen App oder Website nutzen. Möglich ist alles, was sich mit HTML, CSS und Javascript umsetzen lässt, von einer einfachen Überschrift mit fest definiertem Aussehen über einen Videoplayer bis hin zu einem PDF-Tag, das eine entsprechende Datei im Browser mittels pdf.js rendert. Auch komplette Applikationen lassen sich in Form eines solchen Tags einfügen.
%Vieles, was heute über Javascript-Bibliotheken abgewickelt wird, könnte künftig in Form einzelner Webkomponenten umgesetzt werden. Das verringert Abhängigkeiten und sorgt für mehr Flexibilität. Bis die dafür notwendigen Webstandards aber verabschiedet, in Browsern umgesetzt und diese bei ausreichend Nutzern installiert sind, wird aber noch einige Zeit vergehen. Google hat daher mit Polymer eine Bibliothek entwickelt, die die Nutzung von Webkomponenten schon heute ermöglicht und dazu je nach den im Browser vorhandenen Funktionen die fehlenden Teile ergänzt. Damit lassen sich Web-Apps deutlich schneller entwickeln.
%Web-Komponentenbasierte Entwicklung zielt auf die Entkopplung von Web-Anwendungen. Das heißt, dass verwendete Module unabhängig von einander sein sollen und somit die Module wiederverwertbar und anpassbare Software Einheiten darstellen. In diesem Paradigma ist bereits die Trennung von Model, View und Controller vorbehalten. Web Komponenten sind Einheiten, die in sämtlichen Applikationen verwendet werden können und anpassbar an die Voraussetzungen der jeweiligen Applikation sind. Dies ist eine der gängigsten Technologien, um große und komplexe Applikationen, die eine hohe Anforderung an Wartbarkeit haben, umzusetzen.



\subsection{Forschungsfeld und Forschungsfrage}
\label{sec:1_Forschungsfrage}
\todo{Warum werden Web-Components als die Zukunft des Webs gesehen?}
Welche Vor- und Nachteile haben Web-Components, die mit Google Polymer umgesetzt wurden, gegenüber Web-Components, die mit den W3C-Standards umgesetzt wurden?
Um die Forschungsfrage der Arbeit beantworten zu können, müssen zuerst einige Subfragen beantwortet werden, die die Hauptforschungsfrage mit sich bringt und das Forschungsfeld definieren.\\
Wie ist eine klassische Softwarekomponente definiert? Im Zusammenhang mit der Definition einer Softwarekomponente wird der Begriff \glqq Softwarearchitektur\grqq\ genannt, welcher demnach auch geklärt werden muss. Was unterscheidet klassische Softwarekomponenten von Web-Komponenten? Welche Voraussetzungen haben Web-Komponenten? Welche Probleme werden durch Web-Komponenten gelöst?
Das Subfragen des Forschungsfelds werden mit Hilfe von Literatur beantwortet. Danach wird mit Hilfe eines Praxisbeispiels gezeigt, inwiefern die zuvor beantworteten Fragen mit der Entwicklung von Web-Components zusammenhängen.


\subsection{Struktur der Arbeit}
\label{sec:1_Struktur}

