\selectlanguage{english}
\subsection*{Abstract}

The goal of this thesis is to analyse similarities and distinctions between Web-Components and concepts of software-architecture and software-development. Web-Components standardize the development of components on the web. Yet, components were always developed with the help of various frameworks and libraries. Components based on two different frameworks or libraries were not compatible with each other. To use components, which are based on different frameworks or libraries, all used frameworks or libraries had to be integrated as dependency which in turn led to some interferences. The correct use of Web-Components guarantees that a developed component is fully encapsulated and can be used through a defined interface. Therefore there are no interferences with other components and so components are able to interoperate with each other. This thesis clarifies how far the development of components developed with Web-Components allow concepts of software architecture. Specifically Web-Components are compared with the concepts of service-oriented software-architecture, component-based software-architecture and component-based software development. Web-Components is a very young standard (2013) and thereby this technology on behalf of the browsers is not fully supported. The project \glqq Polymer\grqq\ offers polyfills for all functionalities of Web-Components. With that said, the above mentioned concepts are compared with both Web-Components and polymer. Furthermore polymer as a whole is compared with Web-Components in respect of any perspective (supplied functionality, support, etc.).

\paragraph{Keywords:}
Software-Architecture, Software-Components, serviceoriented Software-Architecture, Component-based Software-Architecture, Component-based Software-Engineering, Web-Components, Polymer

\selectlanguage{ngerman}
