\section{Softwarearchitektur und Softwarekomponenten}
\label{sec:2_Architektur}
Das Kapitel \ref{sec:2_Architektur} verhilft der Leserin und dem Leser den Begriff \glqq Softwarearchitektur\grqq\ zu verstehen. Demnach sollen Vorteile von der Verwendung von Softwarearchitektur genannt werden. Auch ist zu erwähnen, dass Softwarearchitektur auch Nachteile bringt, die jedoch nicht in dieser Arbeit behandelt werden. Weiterhin werden wichtige Begriffe wie serviceorientierte Architektur beziehungsweise komponentenbasierte Softwarearchitektur verbunden mit komponentenbasierter Softwareentwicklung näher erläutert.

Zu Beginn wird geklärt, wie eine klassische Softwarekomponente definiert ist. Daraufhin werden diverse Sichtweisen einer Komponente an Hand von Abbildung \ref{fig:Komponente_Sichtweise} auf Seite \pageref{fig:Komponente_Sichtweise} gezeigt. Folglich werden auf Basis der erklärten Definition mehrere Arten von Komponenten aufgelistet. Hierbei wird bereits der Begriff Softwarearchitektur genannt, der im darauffolgenden Kapitel beschrieben wird. Nach einer kurzen, allgemeinen Definition dieses Begriffs, erfolgt die Überleitung und Verbindung von Architektur und Software. Es ist zu erwähnen, dass in diesem Kapitel nur Architektur behandelt wird, die sich über die Erstellung, Auslieferung und den Betrieb von Software jeglicher Art erstreckt. Folglich gibt es Berührungspunkte zu anderen Architektur-Arten wie zum Beispiel der Daten- oder Sicherheitsarchitektur, die jedoch nicht in dieser Arbeit behandelt werden.
Danach werden sowohl die serviceorientierte Architektur, als auch die komponentenbasierte Architektur in Verbindung mit komponentenbasierter Softwareentwicklung erklärt. Als Abschluss dieses Kapitel wird der Unterschied zwischen einem Service und einer Komponente an Hand von mehreren Punkten beschrieben.