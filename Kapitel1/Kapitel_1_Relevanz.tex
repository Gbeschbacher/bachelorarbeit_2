\subsection{Relevanz}
\label{sec:1_Relevanz}

Zu Beginn muss geklärt werden, was eine Softwarekomponente im Allgemeinen definiert. 1996 wurde die Softwarekomponente bei der European Conference on Object-Oriented Programming (ECOOP) folgendermaßen definiert \citereset \autocite[siehe][S. 35-47]{Szyperski.2002}:
\begin{quote}
\glqq A software component is a unit of composition with contractually specified interfaces and explicit context dependencies only. A software component can be deployed independently and is subject to composition by third parties.\grqq
\end{quote}
Um dies näher zu erläutern, wird ein gängiger Tätigkeitsbereich eines Softwarearchitekten herangezogen: das Erstellen einer Liste von Komponenten, die in die gesamte Architektur problemlos eingefügt werden kann. Diese Liste gibt dem Entwicklungsteam vor, welche Komponenten das Softwaresystem zum Schluss umfassen wird. In einfachen Worten zusammengefasst sind Softwarekomponenten die Teile, die eine Software definiert \citereset \autocite[siehe][S. 35-47]{Szyperski.2002}.

Komponentenentwicklung bezeichnet die Herstellung von Komponenten, die eine spezielle Funktion in einer Softwarearchitektur übernehmen. Sämtliche Komponenten sollten immer für sich gekapselt und unabhängig von einander sein. Dies garantiert die Wiederverwendbarkeit von bereits entwickelten Komponenten. Mehrere Komponenten werden mit Hilfe eines Verbindungsverfahren zusammengeführt, beziehungsweise verwendet.

Softwarekomponenten haben ihren Ursprung im \glqq Unterprogramm\grqq . Ein \glqq Unterprogramm\grqq , oder auch \glqq Subroutine\grqq\ genannt, ist der Teil einer Software, die von anderen Programmen beziehungsweise Programmteilen aufgerufen werden kann. Eine Subroutine gilt als Ursprung der ersten Einheit für die Softwarewiederverwendung \citereset \autocite[siehe][]{Wheeler.1985}.

Programmiererinnen und Programmierer entdeckten, dass sie sich auf die Funktionalität von zuvor geschriebenen Codesegmenten berufen können, ohne sich beispielsweise um ihre Implementierung kümmern zu müssen. Neben der Zeitersparnis, die dadurch entstand, erweiterte diese \glqq Technik \grqq\ die Denkweisen: Der Fokus beim Entwickeln kann auf neue Algorithmen und komplexere Themen gelegt werden. Weiterhin entwickelten sich auch die Programmiersprachen weiter \citereset \autocite[siehe][S. 3-12]{Szyperski.2002}.

%THIS NEEDS TO BE REFACTORED
%Komponentenbasierte Softwareentwicklung entwickelte sich in einer ununterbrochenen Linie von diesen frühen Anfängen. Moderne Komponenten, von denen die meisten webbasierte Services sind, sind viel größer und viel anspruchsvoller, als früher.\todo[inline]{Satz umschreiben - siehe buch dafür} Diese Komponenten besitzen meist eine dienstorientierte Architektur, was zu höherer Komplexität bezüglich der Interaktionsmechanismen führt, als die damaligen Subroutinen \citereset \autocite[siehe][]{Andresen.2003}.
%
In der gleichen Weise, wie die frühen Subroutinen die Programmiererinnen und Programmierer vom Nachdenken über spezifische Details befreit haben, verschiebt sich durch komponentenbasierte Softwareentwicklung der Schwerpunkt von direkter Programmiersoftware zu komponierten Softwaresystemen, also komponentenbasierten Systemen. Das heißt, dass der Schwerpunkt sich in Richtung Integration von Komponenten verlagert hat. Darauf basiert die Annahme, dass es genügend Gemeinsamkeiten in  großen Softwaresystemen gibt, um die Entwicklung von wiederverwendbaren Komponenten zu rechtfertigen. Diese Komponenten können dann auf Grund ihrer Gemeinsamkeiten für weitere Systeme genützt werden. Heute werden Komponenten gesucht, die eine große Sammlung von Funktionen bereitstellen. Für Unternehmen werden beispielsweise nicht nur simple Adressbücher, sondern ganze CRM-Systeme\footnote{Ein \glqq Customer Relationship Management System\grqq\ bezeichnet ein System, das zur Kundenpflege beziehungsweise Kundenkommunikation dient.} gesucht, um die Daten beziehungsweise Kommunikation seiner Kunden zu sichern. Darüber hinaus sollte dieses System auch flexibel sein, d.h. es sollte mit anderen Komponenten erweitert werden können \citereset \autocite[siehe][S. 17-25]{Andresen.2003}.

Komponentenbasierte Entwicklung dient des Weiteren der Verwaltung von Komplexität innerhalb eines Systems. Sie versucht die Komplexität gering zu halten, indem die Programmiererin und der Programmierer sich vollständig auf die Implementierung der Komponenten fokussieren können. Das Verknüpfen beziehungsweise Kombinieren von Komponenten sollte demnach nicht mehr Aufgabe der Entwicklerin und des Entwicklers sein. Um diese Aufgabe zu lösen wird ein Architektur-Framework\footnote{Ein Komponenten-Architektur-Framework basiert auf einer Reihe von Plattform Entscheidungen, einer Reihe von Komponenten-Frameworks und ein Interoperabilitätsdesign der Komponenten-Frameworks \citereset \autocite[siehe][419-422]{Szyperski.2002}.} eingesetzt, das Aktivitäten zur Identifikation, Spezifikation, Realisierung und Verteilung von Komponenten beschreibt. Vorteile durch dieses Programmierparadigma sind somit einerseits die Zeitersparnisse und andererseits die erhöhte Qualität der Komponenten \citereset \autocite[siehe][S. 1-3]{Andresen.2003}. Standardmäßig werden in einem Softwaresystem Annahmen über den Kontext impliziert, in dem das System funktioniert. Die komponentenbasierte Softwareentwicklung verlangt,
\begin{quote}
  \glqq dass all diese Annahmen explizit definiert werden, damit das System in verschiedenen Kontexten wiederverwendet werden kann \citereset \autocite[siehe][]{Andresen.2003}.\grqq
\end{quote}

Durch diese explizite Definition von Annahmen gibt es verschiedene Anwendungsszenarien, die automatisch als Testszenarien der Komponenten dienen.

Im Zeitalter von Internet, Intranet und Extranet sind viele verschiedene Systeme und Komponenten miteinander zu verbinden. Aus den folgenden Gründen ist eine erweiterbare und flexible Architektur notwendig, die eine schnelle Reaktion auf neue Anforderungen ermöglicht: Web-basierte Lösungen sollen auf Informationen eines Server zugreifen können, um z.B. mittels eines Extranets diversen Händlern transparente Einblicke in die Lagerbestände eines Unternehmens zu ermöglichen. ERP\footnote{Enterprise Resource Planning}- und CRM-Systeme sollen eingebunden werden, um die Betreuung und den Service für Kunden zu optimieren \citereset \autocite[siehe][S. 39-42]{Andresen.2003}.

Um diesen Anforderungen gerecht zu werden, gab es in den letzten Jahren eine unüberschaubare Anzahl an Frameworks beziehungsweise Bibliotheken, die die Entwicklung von Komponenten vereinfachen. Das Problem hierbei war, dass die Komponenten, die mit Hilfe der Frameworks beziehungsweise Bibliotheken entstanden sind, nicht mit anderen Frameworks beziehungsweise Bibliotheken verknüpft werden konnten. Eine Standardisierung, wie Komponenten im Web-Bereich aussehen müssen beziehungsweise wie die Schnittstellen definiert sein müssen, um Wiederverwendbarkeit garantieren zu können, gab es bis dato nicht. Eine neue Technologie, die zur Zeit vom W3C standardisiert wird, gehört zu den interessantesten Webtechniken, da sie eine Standardisierung für Komponenten im Web-Bereich bietet. Diese Technologie versucht eine Vielzahl von Funktionen der populärsten JavaScript-Komponentenframeworks aufzunehmen und nativ in den Browser zu portieren. Dadurch wird es möglich, benutzerdefinierte Komponenten und Applikationen entwickeln zu können, ohne dabei zahlreiche andere Bibliotheken einbinden zu müssen. Weiters wird durch die Standardisierung die Wiederverwendbarkeit beziehungsweise Interoperabilität von Komponenten gewährleistet.


\subsection{Motivation}
\label{sec:1_Motivation}

Die persönliche Motivation zu diesem Thema entstand grundsätzlich in der Zeit eines Praktikums. Hier war es erforderlich, Front-End Komponenten für verschiedene Web-Applikationen zu entwickeln. Grundanforderung der Komponenten war, dass sie wiederverwendet werden können. Hauptproblem bei der Entwicklung der Komponenten war, dass die Web-Applikationen auf unterschiedlichen Frameworks basierten und sämtliche Komponenten immer an den \glqq Standard\grqq\ der Frameworks angepasst werden mussten.

Web-Components bietet als erste Technologie einen allgemeinen Standard für Komponenten im Web-Bereich, wodurch sie für mich sehr interessant ist. Dadurch, dass Web-Components jedoch noch unter schlechter Browser-Unterstützung leiden, wird in der Bachelorarbeit auch der Polyfill Polymer von Google näher analysiert. Polymer versucht jegliche Funktionen von Web-Components im Browser zu emulieren, um sie somit für sämtliche \glqq Evergreen\grqq -Browser zur Verfügung stellen zu können.

Ein weiterer Punkt meiner persönlichen Motivation ist die grundsätzliche Änderung beziehungsweise Erweiterung einiger Konzepte der Web-Entwicklung durch Web-Components. Durch die vollständige Kapselung von Komponenten können keine Abhängigkeitsprobleme untereinander mehr entstehen. Beispielsweise hierfür sind unterschiedliche jQuery-Versionen unter den Komponenten. Auch ist das erstellen von benutzerdefinierten Elementen eine große Änderung. Das Markup von Komponenten kann zukünftig mittels eines benutzerdefinierten Tags gerendert werden. Dies hilft vor allem bei Komponenten, die sehr darstellungsabhängig sind, sprich viele \lstinline|<div>|-Elemente benötigen.

Auf Grund des bereits zuvor genannten Hauptproblems bei der Entwicklung von Komponenten wird auch ein Praxisprojekt im Zuge der Bachelorarbeit erstellt. Hierbei wird der Fokus auf die Programmierung von zwei wiederverwertbaren Frontend-Oberflächen gelegt. Zum einen wird eine Diagramm-Komponente und zum anderen ein komplexes Menü entwickelt, dass für spätere Projekte weiterverwendet werden kann. Die Hauptanforderung ist einerseits die Kompatibilität zu so vielen Browsern wie möglich zu gewährleisten und andererseits die Interoperabilität mit anderen Komponenten sicherzustellen. Des Weiteren soll durch die einmalige Programmierung dieser Komponente und deren darauffolgende Wiederverwendung der Wartungsaufwand möglichst gering gehalten werden. Somit wird in dieser Arbeit geklärt, inwiefern die Entwicklung von Web-Components ohne Unterstützungen wie beispielsweise das Polymer Projekt bereits möglich ist. Weiterhin soll geklärt werden, welche Aspekte der klassischen Softwarearchitektur bei der Entwicklung von Web-Components aufgegriffen werden und inwiefern es mit komponentenbasierter Softwareentwicklung beziehungsweise komponentenbasierter Softwarearchitektur vereinbar ist. Folgend werden diese Aspekte nicht nur an Hand der zur Zeit standardisierten Technologie namens Web-Components analysiert, sondern auch an Hand des von Google zur Verfügung gestellten Polyfills namens Polymer.
