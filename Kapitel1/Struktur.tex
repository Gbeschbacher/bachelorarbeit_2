
\subsection{Struktur der Arbeit}
\label{sec:1_Struktur}

\todo[inline]{Anpassen (mit neuen Kapitel etc.)}

Die Arbeit beginnt mit einer allgemeinen Einführung zu den Begriffen \glqq Softwarearchitektur\grqq\ und \glqq Softwarekomponenten\grqq\ (siehe Kapitel \ref{sec:2_Softwarekomponente_Klassisch} auf Seite \pageref{sec:2_Softwarekomponente_Klassisch}). Nach dieser Erklärung werden die verschiedenen Arten von Softwarekomponenten aufgezeigt und näher erläutert (siehe Kapitel \ref{sec:2_Arten_Komponenten} auf Seite \pageref{sec:2_Arten_Komponenten}). Folgend wird der bereits erklärte Begriff der Softwarearchitektur (siehe Kapitel \ref{sec:2_Softwarearchitektur} auf Seite \pageref{sec:2_Softwarearchitektur}) auf zwei Teilbereiche beschränkt: einerseits die serviceorientierte Softwarearchitektur (siehe Kapitel \ref{sec:2_Serviceorientierte_Softwarearchitektur} auf Seite \pageref{sec:2_Serviceorientierte_Softwarearchitektur}) und andererseits die komponentenbasierte Softwarearchitektur (siehe Kapitel \ref{sec:2_Komponentenbasierte_Softwarearchitektur} auf Seite \pageref{sec:2_Komponentenbasierte_Softwarearchitektur}). Als Abschluss des Kapitels wird der feine Unterschied zwischen einem Dienst und einer Komponente näher erläutert (siehe Kapitel \ref{sec:2_Unterschied_Dienst_Komponente} auf Seite \pageref{sec:2_Unterschied_Dienst_Komponente}).

Das darauffolgende Kapitel bietet zu Beginn einen Überblick über Web-Components (siehe Kapitel \ref{sec:3_W3C} auf Seite \pageref{sec:3_W3C}). Weiters wird die Spezifikation dieser Technologien an Hand einiger Beispiele näher erläutert (beginnend bei Kapitel \ref{sec:3_WC_Templates} auf Seite \pageref{sec:3_WC_Templates} ff.). Auf Grund der mangelnden Browser-Unterstützung von Web-Components (siehe Kapitel \ref{sec:3_WC_Support} auf Seite \pageref{sec:3_WC_Support}) wird folglich Google-Polymer vorgestellt. Dieses Projekt soll sämtliche \glqq Evergreen\grqq -Browser unterstützen (siehe Kapitel \ref{sec:4_Polymer_Support} auf Seite \pageref{sec:4_Polymer_Support}) und somit die Entwicklung mit Web-Components Technologien bereits ermöglichen. Das Framework wird in Kapitel \ref{sec:4_Polymer} auf Seite \pageref{sec:4_Polymer} genauer erklärt.

In Kapitel \ref{sec:5_Vergleich_WC_Polymer} auf Seite \pageref{sec:5_Vergleich_WC_Polymer} wird ein detaillierter Vergleich von Web-Components und Polymer erstellt. Dieser Vergleich beinhaltet Aspekte beziehungsweise Relationen zur klassischen Softwareentwicklung (siehe Kapitel \ref{sec:5_Vergleich_Architektur} auf Seite \pageref{sec:5_Vergleich_Architektur}), serviceorientierten Softwarearchitektur (siehe Kapitel \ref{sec:5_Vergleich_SOA} auf Seite \pageref{sec:5_Vergleich_SOA}), komponentenbasierten Softwarearchitektur (siehe Kapitel \ref{sec:5_Vergleich_CBA} auf Seite \pageref{sec:5_Vergleich_CBA}) und komponentenbasierten Softwareentwicklung (siehe Kapitel \ref{sec:5_Vergleich_CBSE} auf Seite \pageref{sec:5_Vergleich_CBSE}).

Als Praxisprojekt der Arbeit wurden zwei Komponenten definiert: eine Diagramm-Komponente und eine Menükomponente. Beide Komponenten werden zum einen mit Hilfe von nativen Technologien und zum anderen mit Hilfe von Google-Polymer umgesetzt (siehe Kapitel \ref{sec:6_WC_Pur} auf Seite \pageref{sec:6_WC_Pur} und Kapitel \ref{sec:6_WC_Polymer} auf Seite \pageref{sec:6_WC_Polymer}).

Die bis zu diesem Zeitpunkt definierten Begriffe und Erkenntnisse werden im Abschlusskapitel der Arbeit nochmal aufgegriffen und kurz beschrieben (siehe Kapitel \ref{sec:7_Konklusion} auf Seite \pageref{sec:7_Konklusion}). Abschließend wird der Ausblick sowie noch offene Fragen dieser Arbeit geklärt.