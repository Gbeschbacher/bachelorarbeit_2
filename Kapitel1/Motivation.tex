\subsection{Motivation}
\label{sec:1_Motivation}

Die persönliche Motivation zu diesem Thema entstand grundsätzlich in der Zeit eines Praktikums des Autors. Hier war es erforderlich, Front-End Komponenten für verschiedene Web-Applikationen zu entwickeln. Grundanforderung der Komponenten war, dass sie wiederverwendet werden können. Hauptproblem bei der Entwicklung der Komponenten war, dass die Web-Applikationen auf unterschiedlichen Frameworks basierten und sämtliche Komponenten immer an den \glqq Standard\grqq\ der Frameworks angepasst werden mussten.

Web-Components bietet als erste Technologie einen allgemeinen Standard für Komponenten im Web-Bereich, wodurch sie für mich sehr interessant ist. Dadurch, dass Web-Components jedoch noch unter schlechter Browser-Unterstützung leiden, wird in dieser Arbeit auch der Polyfill Polymer von Google näher analysiert. Polymer versucht jegliche Funktionen von Web-Components im Browser zu emulieren, um sie somit für sämtliche \glqq Evergreen\grqq -Browser zur Verfügung stellen zu können.

Ein weiterer Punkt der persönlichen Motivation des Autors ist die grundsätzliche Änderung beziehungsweise Erweiterung einiger Konzepte der Web-Entwicklung durch Web-Components. Durch die vollständige Kapselung von Komponenten können keine Abhängigkeitsprobleme untereinander mehr entstehen. Beispielsweise hierfür sind unterschiedliche jQuery-Versionen unter den Komponenten. Auch ist das erstellen von Custom-Elements eine große Änderung. Das Markup von Komponenten kann zukünftig mit Hilfe eines benutzerdefinierten Tags gerendert werden. Dies hilft vor allem bei Komponenten, die sehr darstellungsabhängig sind.

Auf Grund des bereits zuvor genannten Hauptproblems bei der Entwicklung von Komponenten wird auch ein Praxisprojekt im Zuge der Bachelorarbeit erstellt. Hierbei wird der Fokus auf die Programmierung von zwei wiederverwertbaren Frontend-Oberflächen gelegt. Zum einen wird eine Diagramm-Komponente und zum anderen ein komplexes Menü entwickelt, dass für spätere Projekte weiterverwendet werden kann. Die Hauptanforderung ist einerseits die Kompatibilität zu so vielen Browsern wie möglich zu gewährleisten und andererseits die Interoperabilität mit anderen Komponenten sicherzustellen. Des Weiteren soll durch die einmalige Programmierung dieser Komponente und deren darauffolgende Wiederverwendung der Wartungsaufwand möglichst gering gehalten werden. Somit wird in dieser Arbeit geklärt, inwiefern die Entwicklung von Web-Components ohne Unterstützungen wie beispielsweise das Polymer Projekt bereits möglich ist. Weiterhin soll geklärt werden, welche Aspekte der klassischen Softwarearchitektur bei der Entwicklung von Web-Components aufgegriffen werden und inwiefern es mit komponentenbasierter Softwareentwicklung beziehungsweise komponentenbasierter Softwarearchitektur vereinbar ist. Folgend werden diese Aspekte nicht nur an Hand der zur Zeit standardisierten Technologie namens Web-Components analysiert, sondern auch an Hand des von Google zur Verfügung gestellten Polyfills namens Polymer.
