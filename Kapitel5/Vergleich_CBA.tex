\subsection{Vergleich hinsichtlich komponentenbasierter Softwarearchitektur}
\label{sec:5_Vergleich_CBA}

Auch in Bezug auf die komponentenbasierte Softwarearchitektur bietet Polymer mehr Hilfestellungen als native Web-Components. Diese Form der Softwarearchitektur versucht sowohl die Eigenschaften der verwendeten Komponenten und Systeme, als auch ihre Abhängigkeiten und Kommunikationsarten für das zu entwickelnde System zu definieren. Um Abhängigkeiten in einem System zu lösen verwenden native Web-Components und Polymer die Technologie von HTML-Imports. Hierbei gibt es keinerlei Unterschiede bei der Verwendung.

Polymer versucht jedoch im Gegensatz zu Web-Components einige Kommunikationsarten in einem System zu regeln. Ein Beispiel in diesem Kontext wäre das \lstinline|<polymer-ajax>|-Element, welches versucht, die Verwendung von \lstinline|XMLHttpRequests| zu standardisieren.
Des Weiteren gibt Polymer gewisse \glqq Coding-Conventions\grqq\ vor, um die Interoperabilität zwischen Komponenten garantieren zu können. Auch sollen die Interferenzen durch die Einhaltung dieser Richtlinien gering gehalten werden. Folglich bietet somit Polymer Hilfestellungen bei \glqq Bau\grqq\ eines Systems beziehungsweise Teilsystems.