\subsection{Offene Fragen hinsichtlich der Entwicklung}
\label{sec:5_Fragen}

In Kapitel \ref{sec:2_Softwarearchitektur} auf Seite \pageref{sec:2_Softwarearchitektur} wird der theoretische Unterschied zwischen serviceorientierter Softwarearchitektur und komponentenbasierter Softwarearchitektur kurz beschrieben. Hinsichtlich der Entwicklung von Web-Components könnte man erforschen, inwiefern sie als Service eingesetzt werden können beziehungsweise welche Anforderungen dafür benötigt werden.

In Kapitel \ref{sec:3_Web_Components} auf Seite \pageref{sec:3_Web_Components} werden lediglich die Konzepte von Web-Components und deren Verwendung näher beschrieben. Jedes Subkapitel könnte dahingehend erweitert werden, dass die dahinterstehende Browser-Funktionalitäten näher erläutert werden.

Auch werden bei der Entwicklung der Komponenten in Kapitel \ref{sec:4_Web_Components_Praxis} auf Seite \pageref{sec:4_Web_Components_Praxis} nur zwei Aspekte betrachtet: native Web-Components und Google-Polymer. Hier könnten mehrere Frameworks beziehungsweise Bibliotheken miteinander verglichen werden. Einerseits könnten sämtliche Polyfills von Web-Components und andererseits sämtliche Bibliotheken, die der Entwicklung von Komponenten dienen, genauer betrachtet und gegenübergestellt werden.