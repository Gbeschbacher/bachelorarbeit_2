\subsection{Vergleich hinsichtlich serviceorientierter Softwarearchitektur}
\label{sec:5_Vergleich_SOA}

Sowohl native Web-Components, als auch polymer bieten keine Vorteile hinsichtlich der Erstellung serviceorientierter Architekturen. Diese Art der Architektur dient der Entwicklung von verteilten Systemen. Hierbei können Systemkomponenten eigenständige Dienste darstellen. Das System selbst kann auf geographisch verteilten Rechnern laufen. Die standardisierten Kommunikationsprotokolle wurden für den Informationsaustausch zwischen Diensten entwickelt. Software-Systeme können durch Komposition lokaler und externer Dienste aufgebaut werden. Diese Dienste wiederum können auf lokal vorhandene Komponenten zurückgreifen. Weder Web-Components, noch Polymer bieten Hilfestellungen hinsichtlich der Kommunikation mit Diensten.
