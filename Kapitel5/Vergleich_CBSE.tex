\subsection{Vergleich hinsichtlich komponentenbasierter Softwareentwicklung}
\label{sec:5_Vergleich_CBSE}

Dadurch, dass Google-Polymer den Ansatz verfolgt, dass \glqq alles eine Komponente ist\grqq\ \citereset \autocite{Polymer}, bietet es hinsichtlich komponentenbasierter Softwareentwicklung Vorteile gegenüber nativen Web-Components. Das zugrunde liegende Paradigma der komponentenbasierten Softwareentwicklung ist, dass ein Software-System auf Basis von Standardkomponenten aufbauen soll, anstatt bereits funktionierende Komponenten neu zu entwickeln. In diesem Kontext stellt Polymer eine Reihe von vordefinierten Komponenten zur Verfügung, was ein klarer Vorteil gegenüber nativen Web-Components ist. Diese Komponenten können als Standardkomponenten angesehen werden, mit der Möglichkeit, sie beliebig erweitern zu können. Auch versuchen diese Standardkomponenten gewisse Funktionen zu standardisieren, wie beispielsweise die Abhandlung eines Ajax-Calls. Native Web-Components hingegen bieten derzeit keine vordefinierten Komponenten an. Somit bietet Polymer Vorteile sowohl bei der Standardisierung betrieblicher und technischer Aufgaben, als auch bei der Entwicklung allgemeiner Komponenten durch Kompositionen von vordefinierten Komponenten.