\subsection{Ausblick von Web-Components}
\label{sec:5_Ausblick}

Die nahe Zukunft von Web-Components ist sehr von der Entwicklung der Browser abhängig. Solange kein einziger Browser Web-Components zu 100\% unterstützt, wird die Akzeptanz dieser Technologien unter den Entwicklerinnen und Entwicklern gering sein. Auch Polyfills helfen hier nur bedingt, da sie nur \glqq Evergreen\grqq -Browser und Internet Explorer 10+ unterstützen. Somit muss noch einige Zeit vergehen, bis die breite Masse die dafür notwendigen Browser installiert haben. Auch ist derzeit noch kein Projekt in Production-Environment bekannt, das auf die Technologien von Web-Components aufbaut.

In ferner Zukunft können Web-Components jedoch die Entwicklung von Komponenten im Web stark verändern. Mit der Akzeptanz der Entwicklerinnen und Entwickler und der Unterstützung der Browser werden Komponenten auf eine neue Art erstellbar sein. Es ist wahrscheinlich, dass JavaScript-Bibliotheken und Komponentenframeworks an Beliebtheit verlieren beziehungsweise auf Technologien von Web-Components umsteigen werden. Darüber hinaus wird es wahrscheinlich eine Plattform geben, die Komponenten bereitstellen wird.