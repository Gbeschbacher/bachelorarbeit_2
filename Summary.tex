\section*{Kurzfassung}
\begin{tabular}{p{4cm} p{12cm}}
Vor- und Zuname:& Georg Eschbacher\\
Institution: & FH Salzburg\\
Studiengang: &  Bachelor MultiMediaTechnology\\
Titel der Bachelorarbeit: & Komponentenbasierte Softwarearchitektur und Softwareentwicklung: \par Ein Vergleich von Web-Components und Google Polymer\\
Begutachter: & Hubert Hölzl, MSc\\
\end{tabular}
\vspace{0.5cm}

Im Rahmen dieser Arbeit werden die Gemeinsamkeiten beziehungsweise Unterschiede zwischen Web-Components und Konzepten der Softwarearchitektur sowie Softwareentwicklung analysiert. Mit Hilfe von Web-Components wird die Entwicklung von Komponenten im Web standardisiert. Bis dato wurden Komponenten immer mit Hilfe diverser Frameworks und Bibliotheken entwickelt. Komponenten, die mit zwei unterschiedlichen Frameworks beziehungsweise Bibliotheken entwickelt wurden, waren nicht interkompatibel. Um Komponenten verwenden zu können, die auf unterschiedlichen Frameworks und Bibliotheken basieren,  mussten sämtliche Frameworks und Bibliotheken als Abhängigkeiten integriert werden, was wiederum zu einigen Interferenzen führte. Durch die korrekte Verwendung von Web-Components wird eine Komponente vollständig gekapselt und kann durch eine definierte Schnittstelle verwendet werden. Somit können einerseits keine Interferenzen zu anderen Komponenten entstehen und andererseits die Komponenten untereinander interoperieren. Inwiefern die Entwicklung von Komponenten mit Hilfe von Web-Components Konzepte der Softwarearchitektur zulassen, wird in dieser Arbeit geklärt. Web-Components werden speziell mit den Konzepten der serviceorientierten Softwarearchitektur, der komponentenbasierten Softwarearchitektur und der komponentenbasierten Softwareentwicklung verglichen. Dadurch, dass Web-Components ein sehr junger Standard ist (2013), wird diese Technologie seitens der Browser noch nicht vollständig unterstützt. Das Projekt \glqq Polymer \grqq\ bietet Polyfills für sämtliche Funktionalitäten von Web-Components an. Somit werden in dieser Arbeit die bereits genannten Konzepte sowohl mit Web-Components, als auch Polymer verglichen. Auch wird Polymer als Ganzes mit Web-Components hinsichtlich sämtlicher Blickwinkel (bereitgestellte Funktionalität, Unterstützung, etc.) verglichen.

\paragraph{Schlagwörter:}
Softwarearchitektur, Softwarekomponenten, serviceorientierte Softwarearchitektur, komponentenbasierte Softwarewarchitektur, komponentenbasierte Softwareentwicklung, Web-Components, Polymer
