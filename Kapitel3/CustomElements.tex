\subsubsection{Custom Elements}
\label{sec:3_WC_Elements}

Custom Elemente sind ein neue Typen von bisher bestehenden DOM-Elementen. Sie können vom Autor beliebig definiert werden und müssen nur wenige Vorschriften einhalten. Im Gegensatz zu Decorators (siehe Kapitel \ref{sec:3_WC_Decorators} auf Seite \pageref{sec:3_WC_Decorators}), welche zustandslos und kurzlebig sind, können Custom Elements den Zustand kapseln und eine Schnittstelle zur Verwendung bereitstellen. Tabelle \ref{tab:Unterschiede} zeigt die Schlüsselunterschiede zwischen den beiden Konzepten.

\begin{table}[h]
\centering
\begin{tabular}{ l || l | l}
& Decorators & Custom Elements \\
\hline
\hline
Lebensdauer & kurzlebig, wenn ein passender CSS-Selector vorhanden ist & stabil, angepasst an die Lebensdauer des Elements\\
\hline
dynamisches hinzufügen, entfernen & Ja, auf Basis des CSS-Selectors & Nein; einmalig (bei der Erstellung des Elements)\\
\hline
In einem Skript erreichbar& Nein; transparent zum DOM und kein Hinzufügen einer Schnittstelle möglich & Ja, im DOM erreichbar und eventuell vorhandene Schnittstelle\\
\hline
Zustand & zustandsloser Ansatz & zustandsorientiertes DOM-Objekt \\
\hline
Behaviour & Simulated by changing decorators & First-class using script and events \\
\end{tabular}
\caption[
Schlüsselunterschiede zwischen Decorators und Custom Elements
]
{Schlüsselunterschiede zwischen Decorators und Custom Elements}
\label{tab:Unterschiede}
\end{table}