\subsubsection{Decorators}
\label{sec:3_WC_Decorators}
Das folgende Kapitel basiert ausschließlich auf der Einführung zu Web-Components aus dem W3C \citereset \autocite{Glakov.2013}.

Decorators sind Elemente, die nach dem Decorator-Pattern benannt sind. Zur Zeit gibt es keinerlei Unterstützung von Seiten der Browser zu diesem Konzept, somit wird in dieser Arbeit das vom W3C definierte Konzept nur theoretisch erläutert. Um jedoch dieses Konzept verstehen zu können muss zuerst das genannte Pattern kurz beschrieben werden.

Grundsätzlich gehört das Decorator-Pattern zu den Struktur-Pattern der Softwareentwicklung. Das Pattern ist eine flexible Alternative zur Unterklassenbildung, um eine Klasse zur Laufzeit um zusätzliche Funktionalitäten erweitern zu können. Hinsichtlich der Verwendung des Patterns und dem Dekorator-Konzept bringt dies kleine Vorteile mit sich. Sie bestehen beispielsweise darin, dass mehrere Dekorierer hintereinandergeschaltet werden können. Weiters ist die Laufzeitbeeinflussung der Dekorierer als Vorteil zu sehen, da sie ausgetauscht werden können. Folglich kann Funktionalität eines individuellen Objekts zur Laufzeit erweitert werden, ohne dabei die Funktionalität von anderen Objekten der selben Klasse zu ändern.

Um Decorators mit Hilfe des W3C-Konzepts näher erklären zu können, wird in diesem Kapitel folgendes Beispiel verwendet. Es gibt eine Liste von Autos, wobei jedes Auto
eine Modellbezeichnung, eine Marke, ein Bild, sowie eine Kurzbeschreibung hat. Das Markup eines Autos würde wie folgt aussehen:

\begin{lstlisting}[language=HTML, caption={Web-Components Decorators - Markup eines Autos}, label={lst:3_Decorators_Basic}, escapeinside={@}{@}]
<li class="car-item">
   <img class="car-image" title="Seat Ibiza" src="images/seat-ibiza.jpg" />
   <h3 class="car-model">Seat Ibiza</h3>
   <p class="car-description">The SEAT Ibiza is a supermini car manufactured by the Spanish automaker SEAT. It is SEAT&#039;s best-selling car and perhaps the most popular model in the company&#039;s range.</p>
</li>
\end{lstlisting}

Unter der Annahme, dass man das Bild des Autos um die Funktionalität erweitern will, sodass man es sichtbar machen, unsichtbar machen beziehungsweise schließen kann, würde man das bereits vorhandene Markup aus Listing \ref{lst:3_Decorators_Basic} auf Seite \pageref{lst:3_Decorators_Basic} wie folgt erweitern:

\begin{lstlisting}[language=HTML, caption={Web-Components Decorators - Markup eines Autos mit Rahmen}, label={lst:3_Decorators_Basic2}, escapeinside={@}{@}]
<li class="car-item">
   <section class="window-frame">
      <header>
          <a class="frame-toggle" href="#">Min/Max</a>
          <a class="frame-close" href="#">Close</a>
      </header>
      <img class="car-image" title="Seat Ibiza" src="images/seat-ibiza.jpg" />
      <h3 class="car-model">Seat Ibiza</h3>
      <p class="car-description">The SEAT Ibiza is a supermini car manufactured by the Spanish automaker SEAT. It is SEAT&#039;s best-selling car and perhaps the most popular model in the company&amp;#039;s range.</p>
   </section>
</li>
\end{lstlisting}

Listing \ref{lst:3_Decorators_Basic2} auf Seite \pageref{lst:3_Decorators_Basic2} würde somit das Markup eines Autos beinhalten, wobei es zwei Buttons gibt: einen zum Umschalten zwischen sichtbar und unsichtbar und einen um das Bild komplett zu löschen.
Wenn man es mit den bisherigen standardisierten Möglichkeiten umsetzt, wird der Quellcode schnell aufgebläht.

Decorators würden in diesem Beispiel bereits helfen. Man könnte definieren, dass spezielle Elemente im DOM mit mehr Markup, Style und zusätzlicher Funktionalität versehen werden. Essentiell hierbei ist, dass es möglich ist die Basisfunktionalität nur für eine gewünschte Menge an Elementen erweitern zu können. Wenn man die erweiterte Funktionalität des Auto-Beispiels mit Hilfe von Decorators umsetzen würde, würde es wie folgt aussehen:

\begin{lstlisting}[language=HTML, caption={Web-Components Decorators - Markup der zusätzlichen Funktionalität (Rahmen)}, label={lst:3_Decorators_Basic3}, escapeinside={@}{@}]
<decorator id="frame-decorator">
   <template>
      <section id="window-frame">
         <header>
            <a id="toggle" href="#">Min/Max</a>
            <a id="close" href="#">Close</a>
         </header>
         @\label{lst:3_Decorators_Basic3_content}@<content></content>
      </section>
   </template>
</decorator>
\end{lstlisting}

Dieses Beispiel bedarf näherer Erklärung. Decorators werden grundsätzlich mit \lstinline|<template>|-Elementen eingesetzt (mehr zu Templates in Kapitel \ref{sec:3_WC_Templates} auf Seite \pageref{sec:3_WC_Templates}). Des Weiteren wird in Zeile \ref{lst:3_Decorators_Basic3_content} der Listing \ref{lst:3_Decorators_Basic3} ein \lstinline|<content>|-Element verwendet. Dies ist zwingend notwendig, da in dieser Stelle der Inhalt des zu dekorierenden Elements eingefügt wird. Auch ist zu erwähnen, dass in diesem Beispiel nur \lstinline|id|s verwendet werden, was nicht zu empfehlen ist. Jedoch sollte es zur Visualisierung dienen, dass \lstinline|id|s innerhalb eines \lstinline|<decorator>|-Elements gekapselt sind. Sie werden nie im DOM erscheinen beziehungsweise verfügbar sein. \lstinline|document.getElementById("window-frame")| wird keine Elemente zurückgeben, weder vor noch nach der Anwendung des \lstinline|<decorator>|-Elements.

Weiterhin ist es möglich, sämtliche Elemente eines Decorators zu gestalten. In Listing \ref{lst:3_Decorators_Basic4} auf Seite \pageref{lst:3_Decorators_Basic4} werden die beiden Buttons mit \lstinline|float: right;| gestaltet. Um die \lstinline|floats| der Elemente wieder zu löschen, wird das \lstinline|<header>|-Element mit der CSS-Klasse \lstinline|clearfix| erweitert.

\begin{lstlisting}[language=HTML, caption={Web-Components Decorators - Markup der zusätzlichen Funktionalität (Rahmen) inklusive Style}, label={lst:3_Decorators_Basic4}, escapeinside={@}{@}, escapechar=!]
<decorator id="frame-decorator">
   <template>
      <section id="window-frame">
        !\colorbox{light-gray}{<style scoped>}!
            !\colorbox{light-gray}{\#toggle {float: right;}}!
            !\colorbox{light-gray}{\#close {float: right;}}!
         !\colorbox{light-gray}{</style>}!
         !\colorbox{light-gray}{<header class="clearfix">}!
            <a id="toggle" href="#">Min/Max</a>
            <a id="close" href="#">Close</a>
         </header>
         <content></content>
      </section>
   </template>
</decorator>
\end{lstlisting}

Es ist zu beachten, dass sämtliche Gestaltungen außerhalb des \lstinline|<style scoped>|-Elements unter Verwendung der richtigen Klassennamen immer noch angewendet werden.

Um nun das bereits vorhandene Markup mit Funktionalität versehen zu können, muss zuerst noch etwas erläutert werden\todo{umschreiben}. Bei Decorators gibt es keine \glqq normalen\grqq\ Events, wie man es gewöhnt ist. Das Hinzufügen beziehungsweise Entfernen eines Decorators würde das Event, wenn es auf ein Element gebunden war, löschen. Anstatt normalen Events hat man mit Hilfe von Decorators die Möglichkeit einen Event-Controller zu erstellen, um mittels diesen Events verwalten zu können.

\begin{lstlisting}[language=HTML, caption={Web-Components Decorators - Markup der zusätzlichen Funktionalität (Rahmen) inklusive Style und Funktionalität}, label={lst:3_Decorators_Basic5}, escapeinside={@}{@}, escapechar=!]
<decorator id="frame-decorator">
   <script>
      this.listen({
         selector:"#toggle", type:"click",
         handler: function (event) {
            // do the toggle button logic here
         }
      });
      this.listen({
         selector:"#close", type:"click",
         handler: function (event) {
            // do the close button logic here
         }
      });
   </script>
   <template>
      <section id="window-frame">
         <style scoped>
            #toggle {float: right;}
            #close {float: right;}
         </style>
         <header class="clearfix">
            <a id="toggle" href="#">Min/Max</a>
            <a id="close" href="#">Close</a>
         </header>
         <content></content>
      </section>
   </template>
</decorator>
\end{lstlisting}


Der in Listing \ref{lst:3_Decorators_Basic5} auf Seite \ref{lst:3_Decorators_Basic5} gezeigte Decorator kann somit verwendet werden, um die gewünschte Funktionalität (das Bild des Autos soll sichtbar, unsichtbar beziehungsweise gelöscht werden können) bereitstellen zu können, ohne dabei für jedes \lstinline|<li>|-Element extra Markup hinzufügen zu müssen. Schlussendlich würde das Beispiel für ein Auto wie folgt aussehen:

\begin{lstlisting}[language=HTML, caption={Web-Components Decorators - Markup eines Autos mit Decorator}, label={lst:3_Decorators_Basic6}, escapeinside={@}{@}, escapechar=!]
<decorator id="frame-decorator">
   <script>
      this.listen({
         selector:"#toggle", type:"click",
         handler: function (event) {
            // do the toggle button logic here
         }
      });
      this.listen({
         selector:"#close", type:"click",
         handler: function (event) {
            // do the close button logic here
         }
      });
   </script>
   <template>
      <section id="window-frame">
         <style scoped>
            #toggle {float: right;}
            #close {float: right;}
         </style>
         <header class="clearfix">
            <a id="toggle" href="#">Min/Max</a>
            <a id="close" href="#">Close</a>
         </header>
         <content></content>
      </section>
   </template>
</decorator>

<li class="car-item">
   <img class="car-image" title="Seat Ibiza" src="images/seat-ibiza.jpg" />
   <h3 class="car-model">Seat Ibiza</h3>
   <p class="car-description">The SEAT Ibiza is a supermini car manufactured by the Spanish automaker SEAT. It is SEAT&#039;s best-selling car and perhaps the most popular model in the company&#039;s range.</p>
</li>
\end{lstlisting}

\begin{lstlisting}[language=CSS, caption={Web-Components Decorators - CSS für die Verwendung von Decorators}, label={lst:3_Decorators_Basic7}, escapeinside={@}{@}, escapechar=!]
.car-item {
  decorator: url(#frame-decorator);
}
\end{lstlisting}

Unter der Verwendung des in Listing \ref{lst:3_Decorators_Basic7} auf Seite \pageref{lst:3_Decorators_Basic7} gezeigten CSS-Attributs wird der Decorator für das in Listing \ref{lst:3_Decorators_Basic6} auf Seite \pageref{lst:3_Decorators_Basic6} verwendet.
