\subsection{Browser Unterstützung}
\label{sec:3_WC_Support}

Die nachstehende Tabelle \ref{tab:BrowserSupport} zeigt die Unterstützung der Browser bezüglich Web-Components. Grün bedeutet, dass die Technologie in dem jeweiligen Browser als stabil gewertet wird und verwendet werden kann. Gelb bedeutet, dass die Technologie vom jeweiligen Browser in Arbeit ist, noch Bugs auftreten, oder mittels eines Flags erreichbar ist. Rot bedeutet, dass es keine Information bezüglich der Technologie in dem jeweiligen Browser gibt und somit auch nicht unterstützt wird.

\begin{description}
\litem{Chrome (Version: 34.0.1847.131)} \hfill \\
Was Web-Components betrifft ist Googles Chrome Vorreiter bei der Implementierung der Spezifikation. Drei Technologien von Web-Components werden bereits als stabil in Chrome bezeichnet. Auch HTML-Importe sind bereits vorhanden, jedoch ist es vor einer Nutzung erforderlich, sie zu aktivieren.

\litem{Opera (Version: 20.0.1387.91)} \hfill \\
Dadurch, dass Opera seit einiger Zeit auf die Basis von Chromium (Blink) gewechselt hat, wird von Opera der gleiche Weg wie von Google erwartet. Zurzeit gibt es keine kennbaren Unterschiede was die Implementierung der Spezifikation bezüglich den beiden Browsern angeht.

\litem{Firefox (Version: 28.0)} \hfill \\
Auch Mozilla versucht mit Firefox den Standard schnellstmöglich umzusetzen, jedoch gibt es einige Bugs diesbezüglich. Nur Templates werden bis dato als stabil angesehen und Custom-Elements, sowie Shadow-DOM sind nur mittels einer Flag erreichbar, beziehungsweise weisen noch Fehler auf.

\litem{Safari (Version: 7.0.3)} \hfill \\
Obwohl viele Funktionen von Web-Components in Webkit implementiert wurden, wurden sie nie in Safari verwendet beziehungsweise zur Verfügung gestellt. Mit der Abzweigung des Chromium-Port von Safari wurde begonnen sämtliche Web-Components Funktionen aus dem Browser zu entfernen.

\litem{Internet Explorer (Version: 11.0.9600.16659)} \hfill \\
Microsoft gibt kein öffentliches Statement bezüglich ihren Entwicklungsplänen ab und somit ist es nicht klar, inwiefern sie die Technologien von Web-Components implementieren werden. Der vor kurzem veröffentlichte Internet Explorer 11 besitzt keine Schnittstellen für Web-Components.

\end{description}

\begin{table}[htbp]
\begin{tabular}{ M{2cm} || M{2cm} | M{2cm} | M{2cm} | M{2cm} | M{2cm} N}
& Chrome & Opera & Firefox & Safari & IE \\
\hline
\hline
Templates & \cRect{green}{1cm}{1cm} & \cRect{green}{1cm}{1cm} & \cRect{green}{1cm}{1cm} & \cRect{red}{1cm}{1cm} & \cRect{red}{1cm}{1cm} &\\[8ex] \hline
HTML-Importe & \cRect{yellow}{1cm}{1cm} & \cRect{yellow}{1cm}{1cm} & \cRect{yellow}{1cm}{1cm} & \cRect{red}{1cm}{1cm} & \cRect{red}{1cm}{1cm} &\\[8ex] \hline
Custom Elements & \cRect{yellow}{1cm}{1cm} & \cRect{yellow}{1cm}{1cm} & \cRect{yellow}{1cm}{1cm} & \cRect{red}{1cm}{1cm} & \cRect{red}{1cm}{1cm} &\\[8ex] \hline
Shadow DOM & \cRect{green}{1cm}{1cm} & \cRect{green}{1cm}{1cm} & \cRect{yellow}{1cm}{1cm} & \cRect{red}{1cm}{1cm} & \cRect{red}{1cm}{1cm} &\\[8ex]
\end{tabular}
\caption[
Browser Unterstützung von Web-Components Úrldate: 13.03.2014
\newline
\small\texttt{http://jonrimmer.github.io/are-we-componentized-yet/}
]
{Browser Unterstützung von Web-Components}
\label{tab:BrowserSupport}
\end{table}

Code-Beispiel \ref{lst:3_Browser_Support} auf Seite \pageref{lst:3_Browser_Support} zeigt, wie die Funktionen beziehungsweise einzelnen Technologien auf Verfügbarkeit im Browser getestet werden können. Das Ergebnis kann in der Konsole des benutzten Browsers eingesehen werden.

\begin{lstlisting}[language=JavaScript, caption={Feature-Detection für Web-Components}, label={lst:3_Browser_Support}, escapeinside={@}{@}]
function supportsTemplate() {
  return 'content' in document.createElement('template');
}
function supportsCustomElements() {
  return 'registerElement' in document;
}
function supportsImports() {
  return 'import' in document.createElement('link');
}
function supportsShadowDom(){
  return typeof document.createElement('div').createShadowRoot === 'function';
}

(function() {
  supportsTemplate()? console.log("Templates are supported!") : console.error("Templates are not supported!")
  supportsCustomElements()? console.log("Custom elements are supported!") : console.error("Custom elements are not supported!")
  supportsImports()? console.log("HTML-Imports are supported!") : console.error("HTML-Imports are not supported!")
  supportsShadowDom()? console.log("Shadow-DOMs are supported!") : console.error("Shadow-DOMs are not supported!")
}());
\end{lstlisting}