
In allgemeinen Worten abgefasst ist Google-Polymer ein Framework, das sämtliche Funktionen von Web-Components zur Verfügung stellt. Polymer verfolgt den Ansatz, dass alles was entwickelt wird, ist eine Web-Komponente und diese entwickelt sich mit dem Web mit. Abbildung \ref{fig:3_polymer_architecture} auf Seite \pageref{fig:3_polymer_architecture} zeigt die Architektur von Polymer.


\begin{figure}[h]
\centering
\includegraphics[height=5.0cm]{images/polymer_architecture.png}
\caption[
Polymers Architektur, Urldate: 04.2014 \newline
\small\texttt{http://i.stack.imgur.com/Ksn6s.png}
]{Polymers Architektur}
\label{fig:3_polymer_architecture}
\end{figure}

\begin{description}
\item[Die rote Schicht] visualisiert die Polyfills, die Polymer zur Verfügung stellt. Diese erlauben die Benutzung von Web-Components. Wichtig hierbei ist, dass die Größe dieser Polyfill-Bibliotheken mit der Weiterentwicklung der Browser abnimmt. Dies bedeutet, dass je mehr Funktionalität von der Spezifikation in den Browsern implementiert ist, desto kleiner sind die Polyfill-Bibliotheken. Der Idealfall für Polymer wäre, dass sämtliche Zusatzbibliotheken, die die nativen Browser-Funktionen emulieren, nicht mehr gebraucht werden.
\item[Die gelbe Schicht] stellt die Meinung von Google dar, wie die spezifizierten Browser Schnittstellen zu Web-Components zusammen verwendet werden sollen. Zusätzlich zu den spezifizierten Technologien werden des Weiteren Fuktionalitäten wie \glqq data-bindings\grqq , \glqq change watcher\grqq , \glqq öffentliche Eigenschaften\grqq , etc.
\item[Die grüne Schicht] repräsentiert eine umfassende Reihe von Interface-Komponenten. Diese entwickelt sich ständig weiter und basieren auf der gelben, sowie roten Schicht.
\end{description}

Polymer bietet die Möglichkeit neben der Erstellung benutzerdefinierter Elemente auch die Verwendung von vordefinierten Elementen. Ein Beispiel für ein vordefiniertes Element wäre das \lstinline|<polymer-ajax>|-Element. Es erscheint in erster Linie als nicht sehr nützlich, jedoch versucht es, einen Standard für Entwickler bereitzustellen, um Ajax-requests zu erstellen beziehungsweise abzuwickeln. Dieses Element ist ähnlich zu folgender Funktion: \lstinline{$.ajax()}\footnote{Mehr Information zur jQuery.ajax-Funktion unter \href{http://api.jquery.com/jQuery.ajax/}{http://api.jquery.com/jQuery.ajax/}}. Der Unterschied zwischen den beiden Möglichkeiten, einen Ajax-Request abzuwickeln, ist, dass die \lstinline{$.ajax()}-Methode Abhängigkeiten besitzt, wohingegen die \lstinline|<polymer-ajax>|-Methode vollkommen unabhängig ist.

Wie man benutzerdefinierte Elemente mittels Polymer entwickelt und welche Coding-Richtlienien dieses Framework beinhaltet, wird in Kapitel \ref{sec:4_WC_Polymer} auf Seite \pageref{sec:4_WC_Polymer} genauer erläutert.